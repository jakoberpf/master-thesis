\cleardoubleoddpage

\chapter*{Kurzfassung}
\thispagestyle{empty} %hide page numbers
Aktuelle Navigationssysteme verwenden häufig Akkumulationsstrategien, um die Reisezeit abzuschätzen, wobei Zeitverzögerungen durch Staus auf der Grundlage der Analyse der Geschichte des zugrunde liegenden Straßennetzes berücksichtigt werden. Dieser Ansatz kann durch ungewöhnliche Ereignisse gestört werden, die kurze Zeitblockaden verursachen, oder durch regelmäßig anfallende, langfristige Verringerungen des Verkehrsaufkommens verzerrt werden. Diese Arbeit evaluiert einen neuen Ansatz zur Vorhersage von Stau- und Unfallmerkmalen durch die Korrelation von Staus und Ereignissen, die in zeitlicher und räumlicher Nähe zueinander liegen. Um dies in einer explorativen Datenanalyse zu evaluieren, werden drei reale Datensätze aus dem Jahr 2019 betrachtet, die Verkehrsbewegungs- und Störfalldaten liefern. Nach einem algorithmischen Ansatz zur Detektion von Staus in FCD Daten und der Lokalisierung räumlich und zeitlich benachbarter Vorfälle aus den BAYSIS und ArbIS Datensätzen wird in der Arbeit mit statistischen Methoden evaluiert, ob und wie diese Vorfälle und Staus miteinander korreliert sind. Daher besteht die Methodik aus dem Clustering von FCD, dem Matching benachbarter Vorfälle, der Korrelations- und Vorhersageanalyse.

Die Ergebnisse zeigen, dass es signifikante Korrelationen zwischen Stau- und Störfallmerkmalen gibt, was bedeutet, dass einzelne Unfallmerkmale statistisch zu Staus mit einer bestimmten Länge und Dauer führen. Dieser Zusammenhang zwischen Länge und Dauer eines Staus ist auch bei den Baustellencharakteristika, wie der Baustellenlage (Straße) und des Baustellenausführungsmonat gegeben. Obwohl diese Korrelationen einen ersten Hinweis auf die Vorhersagbarkeit liefern, ergab eine separate Analyse, dass zwischen keinem der Merkmale Vorhersagbarkeit besteht. An dieser Stelle muss auch angemerkt werden, dass viele Beziehungen nach der Klassifizierung der Daten keine ausreichende Stichprobengröße mehr hatten und ein größerer Datensatz notwendig wäre, um mehr aussagekräftigere und zuverlässigere Ergebnisse zu finden.