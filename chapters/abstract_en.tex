\cleardoubleoddpage

\chapter*{Abstract}
\thispagestyle{empty} %hide page numbers
Current navigation systems often use accumulation strategies to estimate travel time while considering time delays through congestions, which are based on analyzing the history of the time delays on the considered street network. This approach can be disturbed through uncommon events creating short time blockages or be biased through regular accruing, long-term traffic volume reductions. This thesis evaluates a new idea of predicting time delays and accident probability through analyzing the correlation of congestions and incidents which are placed in timely and spatial vicinity of each other. To evaluate this, three real world datasets from the year 2019 will be considered. After an algorithmic approach to analyzing a derivative of a floating car dataset for congestions and locating spatial and timely adjacent incidents from the Bavarian street information systems, the thesis will evaluate if and how these incidents and jams are correlated with each other.