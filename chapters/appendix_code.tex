% -------------------------------------
% -------------------------------------
% ------- Source Code Appendix --------
% -------------------------------------
% -------------------------------------
\chapter{Source Code}

% -----------------------------------
% ------- Cluster Algorithm ---------
% -----------------------------------
\begin{lstlisting}[basicstyle=\tiny, style=py, caption={Compute Correlation Function}, label=lst:compute_correlation] 
\end{lstlisting}

% -----------------------------------
% ------- Shaping Algorithm ---------
% -----------------------------------
\begin{lstlisting}[basicstyle=\tiny, style=py, caption={Compute Correlation Function}, label=lst:compute_correlation] 
\end{lstlisting}

% ------------------------------------
% ------- Matching Algorithm ---------
% ------------------------------------
\begin{lstlisting}[basicstyle=\tiny, style=py, caption={Compute Correlation Function}, label=lst:compute_correlation] 
\end{lstlisting}

% -----------------------------------
% ------- Processing Export ---------
% -----------------------------------
\begin{lstlisting}[basicstyle=\tiny, style=py, caption={Compute Correlation Function}, label=lst:compute_correlation] 
\end{lstlisting}

% ----------------------------------------
% ------- Processing Correlation ---------
% ----------------------------------------
\begin{lstlisting}[basicstyle=\tiny, style=py, caption={Implementation of \textit{compute correlations}}, label=lst:compute_correlation] 
    """
    Calculate the correlation/strength-of-association of features in data-set
    with both categorical and continuous features using:
     * Pearson's R for continuous-continuous cases
     * Correlation Ratio or Point Biserial for categorical-continuous cases
     * Cramer's V or Theil's U for categorical-categorical cases
    
    Parameters:
    -----------
    dataset : NumPy ndarray / Pandas DataFrame
        The data-set for which the features' correlation is computed
    nominal_columns : string / list / NumPy ndarray
        Names of columns of the data-set which hold categorical values. Can
        also be the string 'all' to state that all columns are categorical,
        'auto' (default) to try to identify nominal columns, or None to state
        none are categorical
    mark_columns : Boolean, default = False
        if True, output's columns' names will have a suffix of '(nom)' or
        '(con)' based on there type (eda_tools or continuous), as provided
        by nominal_columns
    continuous_continuous : default is 'pearson', but other correlation
        coefficients can be chosen.
    continuous_nominal : default is 'correlation_ratio', but other correlation
        coefficients can be chosen.
    continuous_dichotomous : default is  'point_biserial', but other correlation
        coefficients can be chosen.
    continuous_ordinal : default is 'spearman', but other correlation
        coefficients can be chosen.
    categorical_categorical : default is 'cramer', but other correlation
        coefficients can be chosen.
    clustering : Boolean, default = False
        If True, hierarchical clustering is applied in order to sort
        features into meaningful groups
    bias_correction : Boolean, default = True
        Use bias correction for Cramer's V from Bergsma and Wicher,
        Journal of the Korean Statistical Society 42 (2013): 323-328.
    nan_strategy : string, default = 'replace'
        How to handle missing values: can be either 'drop_samples' to remove
        samples with missing values, 'drop_features' to remove features
        (columns) with missing values, or 'replace' to replace all missing
        values with the nan_replace_value. Missing values are None and np.nan.
    nan_replace_value : any, default = 0.0
        The value used to replace missing values with. Only applicable when
        nan_strategy is set to 'replace'
    
    Returns:
    --------
    A DataFrame of the correlation/strength-of-association between all features
    """
    dataset = convert(dataset, 'dataframe')  # TODO implement encoding
    
    if nan_strategy == _REPLACE:
        dataset.fillna(nan_replace_value, inplace=True)
    elif nan_strategy == _DROP_SAMPLES:
        dataset.dropna(axis=0, inplace=True)
    elif nan_strategy == _DROP_FEATURES:
        dataset.dropna(axis=1, inplace=True)
    
    columns = dataset.columns
    
    if columns_nominal is None:
        columns_nominal = list()
    elif columns_nominal == 'all':
        columns_nominal = columns
    elif columns_nominal == 'auto':
        columns_nominal = identify_nominal_columns(dataset)
    
    if columns_dichotomous is None:
        columns_dichotomous = list()
    elif columns_dichotomous == 'all':
        columns_dichotomous = columns
    elif columns_dichotomous == 'auto':
        columns_dichotomous = identify_dichotomous_columns(dataset, columns_nominal)
    
    if columns_ordinal is None:
        columns_ordinal = list()
    elif columns_ordinal == 'all':
        columns_ordinal = columns
    elif columns_ordinal == 'auto':
        columns_ordinal = identify_ordinal_columns(dataset, columns_nominal, columns_dichotomous)
    
    corr = pd.DataFrame(index=columns, columns=columns)
    sign = pd.DataFrame(index=columns, columns=columns)
    coef = pd.DataFrame(index=columns, columns=columns)
    
    columns_single_value = []
    
    inf_nan_corr = pd.DataFrame(data=np.zeros_like(corr),
                                columns=columns,
                                index=columns)
    
    inf_nan_sign = pd.DataFrame(data=np.zeros_like(corr),
                                columns=columns,
                                index=columns)
    
    for c in columns:
        # Test if column only contains single value
        if dataset[c].unique().size == 1:
            # Column only contains a single value, prepare for no calculation to be done
            columns_single_value.append(c)
    
    for i in range(0, len(columns)):
    
        if columns[i] in columns_single_value:
            # If column only contains a single value, not correlation calculation necessary
            corr.loc[:, columns[i]] = 0.0
            corr.loc[columns[i], :] = 0.0
            continue
    
        for j in range(i, len(columns)):
    
            if columns[j] in columns_single_value:
                continue
    
            elif i == j:
                # Correlation to itself is always 1.0
                corr.loc[columns[i], columns[j]] = 1.0
                sign.loc[columns[i], columns[j]] = _SIGN_NAN
                inf_nan_sign.loc[columns[i], columns[j]] = _inf_nan_str(_SIGN_NAN)
    
            else:
                # print('Processing ' + columns[i] + ' and ' + columns[j])
                if columns[i] in columns_nominal or columns[i] in columns_dichotomous or columns[i] in columns_ordinal:
                    # i is categorical
                    if columns[j] in columns_nominal or columns[j] in columns_dichotomous or columns[
                        j] in columns_ordinal:
                        # i and j are categorical
                        if theils:
                            # Because Theil's U is asymmetrical, calculate both directions separately
                            ij, p, c = theils_u(dataset[columns[i]], dataset[columns[j]])  # TODO handle two p values
                            ji, p, c = theils_u(dataset[columns[j]], dataset[columns[i]])  # TODO handle two p values
                        else:
                            cell, p, c = cramers_v(dataset[columns[i]], dataset[columns[j]],
                                                   bias_correction=bias_correction)
                            ij = cell
                            ji = cell
                    else:
                        # i is categorical, j is continuous
                        if columns[i] in columns_ordinal:
                            # i is ordinal, j is continuous
                            cell, p, c = kendall(dataset[columns[j]], dataset[columns[i]])
                        elif columns[i] in columns_dichotomous:
                            # i is dichotomous, j is continuous
                            cell, p, c = point_biserial(dataset[columns[j]], dataset[columns[i]])
                        else:
                            # i is nominal, j is continuous
                            cell, p, c = eta(dataset[columns[j]], dataset[columns[i]])
    
                        ij = cell
                        ji = cell
    
                elif columns[j] in columns_nominal or columns[j] in columns_dichotomous or columns[
                    j] in columns_ordinal:
                    # j is categorical, i is continuous
                    if columns[j] in columns_ordinal:
                        # j is ordinal, i is continuous
                        cell, p, c = kendall(dataset[columns[i]], dataset[columns[j]])
    
                    elif columns[j] in columns_dichotomous:
                        # j is dichotomous, i is continuous
                        cell, p, c = point_biserial(dataset[columns[i]], dataset[columns[j]])
                    else:
                        # j is nominal, i is continuous
                        cell, p, c = eta(dataset[columns[i]], dataset[columns[j]])
    
                    ij = cell
                    ji = cell
    
                else:
                    # i and j are continuous
                    assert columns[i] not in columns_nominal or columns[i] not in columns_dichotomous or columns[
                        i] not in columns_ordinal, columns[i] + ' should not be here'
                    assert columns[j] not in columns_nominal or columns[j] not in columns_dichotomous or columns[
                        j] not in columns_ordinal, columns[j] + ' should not be here'
                    cell, p, c = pearsons(dataset[columns[i]], dataset[columns[j]])
    
                    ij = cell
                    ji = cell
    
                corr.loc[columns[i], columns[j]] = round(ij, 2) if not np.isnan(ij) and abs(ij) < np.inf else 0.0
                corr.loc[columns[j], columns[i]] = round(ji, 2) if not np.isnan(ji) and abs(ji) < np.inf else 0.0
                sign.loc[columns[i], columns[j]] = round(p, 4) if not np.isnan(p) and abs(p) < np.inf else _SIGN_NAN
                sign.loc[columns[j], columns[i]] = round(p, 4) if not np.isnan(p) and abs(p) < np.inf else _SIGN_NAN
                coef.loc[columns[i], columns[j]] = c
                coef.loc[columns[j], columns[i]] = c
                inf_nan_corr.loc[columns[i], columns[j]] = _inf_nan_str(ij)
                inf_nan_corr.loc[columns[j], columns[i]] = _inf_nan_str(ji)
                inf_nan_sign.loc[columns[i], columns[j]] = _inf_nan_str(p)
                inf_nan_sign.loc[columns[j], columns[i]] = _inf_nan_str(p)
    
    corr.fillna(value=np.nan, inplace=True)
    sign.fillna(value=np.nan, inplace=True)
\end{lstlisting}

% Cramer's V
\begin{lstlisting}[basicstyle=\tiny, style=py, caption={Compute Correlation Function}, label=lst:compute_correlation]
    def cramers_v(x,
              y,
              bias_correction=True,
              nan_strategy=_REPLACE,
              nan_replace_value=_DEFAULT_REPLACE_VALUE):
        """
        Calculates Cramer's V statistic for categorical-categorical association.
        This is a symmetric coefficient: V(x,y) = V(y,x) uses correction from Bergsma and Wicher, Journal of the Korean Statistical Society 42 (2013): 323-328

        Original function taken from: https://stackoverflow.com/a/46498792/5863503
        Wikipedia: https://en.wikipedia.org/wiki/Cram%C3%A9r%27s_V

        chi2 = ss.chi2_contingency(confusion_matrix)[0]
        n = confusion_matrix.sum()
        phi2 = chi2 / n
        r, k = confusion_matrix.shape
        temp = phi2 - ((k - 1) * (r - 1)) / (n - 1)
        phi2corr = max(0, temp)
        rcorr = r - ((r - 1) ** 2) / (n - 1)
        kcorr = k - ((k - 1) ** 2) / (n - 1)
        return np.sqrt(phi2corr / min((kcorr - 1), (rcorr - 1)))

        Parameters:
        -----------
        x : list / NumPy ndarray / Pandas Series
            A sequence of categorical measurements
        y : list / NumPy ndarray / Pandas Series
            A sequence of categorical measurements
        bias_correction : Boolean, default = True
            Use bias correction from Bergsma and Wicher,
            Journal of the Korean Statistical Society 42 (2013): 323-328.
        nan_strategy : string, default = 'replace'
            How to handle missing values: can be either 'drop' to remove samples
            with missing values, or 'replace' to replace all missing values with
            the nan_replace_value. Missing values are None and np.nan.
        nan_replace_value : any, default = 0.0
            The value used to replace missing values with. Only applicable when
            nan_strategy is set to 'replace'.

        Returns:
        --------
        float : in the range of [0,1]
        float : p-value, calculated with chi-squared
        str   : correlation name/identifier
        """

        print(x.name + ' to ' + y.name + ' with Cramers V')

        if nan_strategy == _REPLACE:
            x, y = replace_nan_with_value(x, y, nan_replace_value)
        elif nan_strategy == _DROP:
            x, y = remove_incomplete_samples(x, y)

        contingency = pd.crosstab(x, y)
        c, p, dof, expected = ss.chi2_contingency(contingency)

        confusion_matrix = pd.crosstab(x, y)
        chi2 = ss.chi2_contingency(confusion_matrix)[0]
        n = confusion_matrix.sum().sum()
        phi2 = chi2 / n
        r, k = confusion_matrix.shape
        if bias_correction:
            phi2corr = max(0, phi2 - ((k - 1) * (r - 1)) / (n - 1))
            rcorr = r - ((r - 1) ** 2) / (n - 1)
            kcorr = k - ((k - 1) ** 2) / (n - 1)
            if min((kcorr - 1), (rcorr - 1)) == 0:
                warnings.warn(
                    "Unable to calculate Cramer's V using bias correction. Consider using bias_correction=False",
                    RuntimeWarning)
                return np.nan
            else:
                return np.sqrt(phi2corr / min((kcorr - 1), (rcorr - 1))), p, 'Cramer\'s V'
        else:
            return np.sqrt(phi2 / min(k - 1, r - 1)), p, r'$V$'
\end{lstlisting}

% Theil's U
\begin{lstlisting}[basicstyle=\tiny, style=py, caption={Compute Correlation Function}, label=lst:compute_correlation]
    def theils_u(x,
             y,
             nan_strategy=_REPLACE,
             nan_replace_value=_DEFAULT_REPLACE_VALUE):
        """
        Calculates Theil's U statistic (Uncertainty coefficient) for categorical-
        categorical association. This is the uncertainty of x given y: value is
        on the range of [0,1] - where 0 means y provides no information about
        x, and 1 means y provides full information about x.

        This is an asymmetric coefficient: U(x,y) != U(y,x)

        Wikipedia: https://en.wikipedia.org/wiki/Uncertainty_coefficient

        s_xy = conditional_entropy(x, y)
        x_counter = Counter(x)
        total_occurrences = sum(x_counter.values())
        p_x = list(map(lambda n: n / total_occurrences, x_counter.values()))
        s_x = ss.entropy(p_x)
        if s_x == 0:
            return 1
        else:
            return (s_x - s_xy) / s_x

        Parameters:
        -----------
        x : list / NumPy ndarray / Pandas Series
            A sequence of categorical measurements
        y : list / NumPy ndarray / Pandas Series
            A sequence of categorical measurements
        nan_strategy : string, default = 'replace'
            How to handle missing values: can be either 'drop' to remove samples
            with missing values, or 'replace' to replace all missing values with
            the nan_replace_value. Missing values are None and np.nan.
        nan_replace_value : any, default = 0.0
            The value used to replace missing values with. Only applicable when
            nan_strategy is set to 'replace'.

        Returns:
        --------
        float : in the range of [0,1]
        float : p-value, calculated with chi-squared
        str   : correlation name/identifier
        """

        print(x.name + ' to ' + y.name + ' with Theils U')

        if nan_strategy == _REPLACE:
            x, y = replace_nan_with_value(x, y, nan_replace_value)
        elif nan_strategy == _DROP:
            x, y = remove_incomplete_samples(x, y)

        contingency = pd.crosstab(x, y)
        c, p, dof, expected = ss.chi2_contingency(contingency)

        s_xy = conditional_entropy(x, y)
        x_counter = Counter(x)
        total_occurrences = sum(x_counter.values())
        p_x = list(map(lambda n: n / total_occurrences, x_counter.values()))
        s_x = ss.entropy(p_x)
        if s_x == 0:
            return 1, 0
        else:
            return (s_x - s_xy) / s_x, p, r'$U$' 

    def conditional_entropy(x,
        y,
        nan_strategy=_REPLACE,
        nan_replace_value=_DEFAULT_REPLACE_VALUE,
        log_base: float = math.e):
        """
        Calculates the conditional entropy of x given y: S(x|y)
        Used by the Theil's U implementation

        Wikipedia: https://en.wikipedia.org/wiki/Conditional_entropy

        Parameters:
        -----------
        x : list / NumPy ndarray / Pandas Series
        A sequence of measurements
        y : list / NumPy ndarray / Pandas Series
        A sequence of measurements
        nan_strategy : string, default = 'replace'
        How to handle missing values: can be either 'drop' to remove samples
        with missing values, or 'replace' to replace all missing values with
        the nan_replace_value. Missing values are None and np.nan.
        nan_replace_value : any, default = 0.0
        The value used to replace missing values with. Only applicable when
        nan_strategy is set to 'replace'.
        log_base: float, default = e
        specifying base for calculating entropy. Default is base e.

        Returns:
        --------
        float
        """
        if nan_strategy == _REPLACE:
        x, y = replace_nan_with_value(x, y, nan_replace_value)
        elif nan_strategy == _DROP:
        x, y = remove_incomplete_samples(x, y)
        y_counter = Counter(y)
        xy_counter = Counter(list(zip(x, y)))
        total_occurrences = sum(y_counter.values())
        entropy = 0.0
        for xy in xy_counter.keys():
        p_xy = xy_counter[xy] / total_occurrences
        p_y = y_counter[xy[1]] / total_occurrences
        entropy += p_xy * math.log(p_y / p_xy, log_base)
        return entropy
\end{lstlisting}

% Eta
\begin{lstlisting}[basicstyle=\tiny, style=py, caption={Compute Correlation Function}, label=lst:compute_correlation]
    def eta(measurements,
        categories,
        nan_strategy=_REPLACE,
        nan_replace_value=_DEFAULT_REPLACE_VALUE):
        """
        Calculates the Correlation Ratio (sometimes marked by the greek letter Eta)
        for categorical-continuous association.

        Answers the question - given a continuous value of a measurement, is it
        possible to know which category is it associated with?

        Value is in the range [0,1], where 0 means a category cannot be determined
        by a continuous measurement, and 1 means a category can be determined with
        absolute certainty.

        Wikipedia: https://en.wikipedia.org/wiki/Correlation_ratio

        Parameters:
        -----------
        measurements : list / NumPy ndarray / Pandas Series
            A sequence of continuous measurements
        categories : list / NumPy ndarray / Pandas Series
            A sequence of categorical measurements
        nan_strategy : string, default = 'replace'
            How to handle missing values: can be either 'drop' to remove samples
            with missing values, or 'replace' to replace all missing values with
            the nan_replace_value. Missing values are None and np.nan.
        nan_replace_value : any, default = 0.0
            The value used to replace missing values with. Only applicable when
            nan_strategy is set to 'replace'.

        Returns:
        --------
        float : correlation coefficient in the range of [0,1]
        float : p-value, calculated with Kruskal-Wallis H (Chi-squared)
        str   : correlation name/identifier
        """
        print(categories.name + ' to ' + measurements.name + ' with Correlation Ration (Eta) and Kruskal-Wallis H')
        if nan_strategy == _REPLACE:
            categories, measurements = replace_nan_with_value(
                categories, measurements, nan_replace_value)
        elif nan_strategy == _DROP:
            categories, measurements = remove_incomplete_samples(
                categories, measurements)
        categories = convert(categories, 'array')
        measurements = convert(measurements, 'array')
        fcat, _ = pd.factorize(categories)
        cat_num = np.max(fcat) + 1
        y_avg_array = np.zeros(cat_num)
        n_array = np.zeros(cat_num)
        for i in range(0, cat_num):
            cat_measures = measurements[np.argwhere(fcat == i).flatten()]
            n_array[i] = len(cat_measures)
            y_avg_array[i] = np.average(cat_measures)
        y_total_avg = np.sum(np.multiply(y_avg_array, n_array)) / np.sum(n_array)
        numerator = np.sum(
            np.multiply(n_array, np.power(np.subtract(y_avg_array, y_total_avg), 2)))
        denominator = np.sum(np.power(np.subtract(measurements, y_total_avg), 2))
        if numerator == 0:
            correlation = 0.0
        else:
            correlation = np.sqrt(numerator / denominator)

        f_statistic, p_value = kruskal_wallis(measurements, categories)
        return correlation, p_value, r'$\eta$'
\end{lstlisting}