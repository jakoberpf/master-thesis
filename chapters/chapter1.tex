\chapter{Introduction}
\setcounter{page}{1}
Traffic \glspl{jam} are a common problem to everyone, ever attempting to begin the summer season with a car trip on the first day of summer break. When the number of road users increases or the capacity of the road way decreases due to various reasons, a imbalance between demand and supply is created \parencite{Tang2019}. This impacts passenger traffic as well as transportation of goods through blocked bottlenecks and decreased travel speeds. They lead to unreliable travel times, inefficient usage of resources, and an increase in emissions, like pollutants or noise \parencite{FHA2011}. Another effect is a decrease in road safety, due to high driver tempers or inattentive mind, which can result in higher accident counts \parencite{Sun2016}. This induces enormous social costs due to billions of hours lost in \glspl{jam} and induced mental stress. \parencite{RetallackOstendorf2019,BardtFritsch2014,ADAC2019}

Therefore it is essential to reduce risks of \glspl{jam}, as well as accidents, with an increased understanding of the traffic accident causes, trigger effects of \glspl{jam} and roadwork consequences, to maintain a fluent traffic flow and traffic safety. Tailored \acrfull{tsm} strategies, focused on automatic reactions for significant traffic event, could enable \acrfull{atdm} of high traffic demands in reduced traffic volume areas \parencite{Tang2019}. This would go towards reducing the economic, environmental, and social costs associated with accidents, roadworks or \glspl{jam}. Part of these \acrshort{tsm} strategies implemented in a \acrshort{atdm} system, could be a form of traffic incident prediction systems, with the potential to identify compromising conditions in real-time, allowing according to actions to be taken to avoid consecutive events. \parencite{RetallackOstendorf2019} 

\bigskip

This thesis approaches \gls{jam} and incident prediction by evaluating the statistical relation of \glspl{jam} and incidents to predict the chance of a consecutive event. These consecutive events can be \glspl{jam}, as well as incidents. Depending on the severity of an accident \glspl{jam} can be provoked. On the other hand \glspl{jam} can facilitate accidents due to the change of traffic flow. Another scenario are construction sites and maintenance which can also lead to both \glspl{jam} and accidents, because of the reduction of traffic volume, changes in road guidance, or other modifications to the actual traffic situation. Although the scope of the thesis does not cover the specifics on a complete production system for \gls{jam} and accident detection, prediction and response, it will take the concept of such a system and focus the possibility of predicting such events, which then would make the development of such a system possible. 

A system capable of the described mentioned functionalities would likely consist of the following processing components.

\begin{itemize}
  \item Acquisition of traffic movement and incident data
  \item Congestion and incident detection
  \item Prediction of consecutive events
  \item Traffic management and controlling responses
\end{itemize}

The second component in charge of detecting \glspl{jam} as well as incident, requires input data like speed, volume, occupancy to represent the traffic situation and incident reports to define incident characteristics. The next component would then analyze the resulting dataset to find characteristic features of the jams and incidents, which will be determined in this thesis. In case the analysis of the characteristic shows a possible and imminent event it would trigger the last component to initiate appropriate controlling actions and prepare incident responses. In the following chapters of this introduction section, the reader will be introduced to the concepts and systems used to cover the input and output requirements in this thesis.

\section{Continuous Floating Data}
\label{introduction_continuous_floating_data}
To detect \glspl{jam} continuous data about the speed or movement of the vehicle on the road is necessary. This kind of information can be collected through a variety of different systems to represent a real-time or at least current picture of the traffic situation. 

The current street network of Bavaria heavily depends on stationary sensors to assess the traffic situation. This includes inductive loops, infrared or radar detectors which can provide traffic indicators like volume, speed, time gaps, jams, density and many others. The data collected with just stationary sensors can only describe the traffic trends restricted to their location and coverage which requires complex simulations and modeling to aggregate data for the missing areas where no sufficient coverage can be provided. Adding to this is the fluctuating result set quality which depends on the input data and simulation model quality. Especially highways are equipped with stationary sensors but the lower index streets network is only covered by a fragmented net of detectors with distances of up to 100\,km between detectors \parencite{INDRIX2015}. Fortunately, nowadays cars as well as drivers and riders are equipped with technology that allows real time tracking and comprehensive data collection. Automobiles can gather information from the build in sensors as well as the on-board GPS. Even mobile devices from drivers and riders can be used to collect location and movement data. \parencite{Randelhoff2016}

\acrfull{fcd} is continuously collected during the usage of a car by the on-board GPS and represents the individual movements. Typically this incorporates the coordinates, timestamp, road section, course and routing data points. These are regularly sent to a central FCD unit/service via mobile radio communication (\acrshort{gsm}-, \acrshort{umts}- or \acrshort{lte}-based) to be aggregated and combined with stationary data to an area wide picture of the traffic situation. In this form they can be used for traffic analysis and management. \parencite{Randelhoff2016,LAPID2020}

A derivative of \acrshort{fcd} is \acrfull{xfcd} developed by BMW. It expands the collection of data points of an \acrshort{fcd} with data from the vehicle sensors and systems like breaks, rain sensors, driver assistance systems and more. These data points add a number of analytic opportunities to generate a more precise and detailed traffic picture. \parencite{LAPID2020}

In contrast to \acrshort{fcd}, \acrlong{fpd} does not need an on-board GPS or car systems to create movement data but assumes that driver’s and rider’s mobile devices will register and deregister at the radio tower along the road. \acrshort{fpd} uses this registration information to determine the radio cell the phone is currently in, how long is stayed in that cell and tracks the devices when switching to another cell or tower. It is therefore able to collect a much larger quantity of datasets but lacks in the accuracy of \acrshort{fcd} which transmit the GPS location of the car itself. That been said, on roads with a high coverage of radio tower like in urban areas or on highways, \acrshort{fpd} is able to achieve a comparable accuracy. Mobile service providers collect this anonymized \acrshort{fpd} and forward them to a traffic management unit, which can analyze the date for disturbances and give feedback through the traffic information channels. \parencite{Randelhoff2016,LAPID2020}

Another type of floating data type is \acrfull{fco}. \acrshort{fco} does not only collect it's own \acrshort{fcd} but also data about it's surroundings with the built-in sensors. This includes the automatic recognition of cross- or opposite traffic, traffic volume or relative speeds to other cars. This additional data not only add detail but also allow for correctional fusion algorithms to reduce uncertainties or errors in the pure \acrshort{fcd}. \parencite{Randelhoff2016}

\section{Street Information Systems}
\label{introduction_street_information_systems}

Besides of having a real-time or current picture of the traffic situation, the concept of the thesis also relies on information about current disturbances or possible triggers of disturbances. 

\subsubsection{\acrfull{baysis}}
For disturbances in form of accidents the \acrfull{baysis}, a publicly available information system from the Bavarian street administration, will be used. The systems task is the acquisition, collection and analysis of street network related information, which contains infrastructure inventory and condition, traffic volumes and other key values, as well as an accident register with detailed reports. An export of this accident register with detailed reports is provided by the \acrfull{zvm} for this thesis.

\subsubsection{\acrfull{arbis}}
Another type of disturbance to consider is roadworks, for which an export from the \acrfull{arbis}, a software tool and database used by the Bavarian infrastructure ministry, will be consulted. The system is used to collect and archive all current, planned and passed roadwork and maintenance projects on the Bavarian state street net-work. With the collected information from \acrshort{arbis}, an effective, economic and safe execution of roadworks can be achieved. Furthermore \acrshort{arbis} can provide detailed report exports about current projects to the Bavarian traffic information office and traffic information channels. \parencite{trafficon2017}

\section{Traffic Status Information}
\label{introduction_traffic_status_information}
To deliver information from traffic management systems to the road users traffic messaging channels can be used. Three known examples are \acrfull{rtti}, \acrfull{tmc} and \acrfull{atis}.

The \acrshort{tmc} is a messenger system for jams and other traffic incidents. The public available service is free to use and publishes current congestion notifications in compatible navigation system and announcements in the local radio channels. In the scope of the \acrshort{tmc} the road network is split into \acrshort{tmc} sections. These sections are used to define the spatial location of the incident to report and typical start or end with road linkups. More detailed Incident information is obtained from the police or reports from traffic participants which adds a considerable before publication delay. It is therefore spatial and temporal to rather imprecise. \parencite{LAPID2020}

Another traffic information source is the \acrshort{rtti}. \acrshort{rtti} supplies traffic participants with information about current events or suggested diversions like \acrfull{tmc}, but with a much spatial higher accuracy. Through a \textit{Geocast}, which is an expansion of a multicast with a geolocation, the spatial precision of the \acrshort{rtti} is superior to the \acrshort{tmc}. This \textit{Geocast} can be either a geometrical address like a GSM84 coordinate or a symbolic address, reaching a spatial accuracy of up to 100m.  \parencite{LAPID2020,HindenDeering2006,ImielinskiNavas1996}. This is why \acrshort{rtti} is the industry standard of suppling vehicle and third party navigation system with up to date traffic information. Another difference to \acrshort{tmc} is the accessibility. Unlike the publicly available \acrshort{tmc}, \acrshort{rtti} is vendor specific and most of the time a payed service, like BMWs ConnectedDrive \parencite{BMW2020}. 

% TODO more infos and literature to TMC / Geocast
% https://ec.europa.eu/transport/themes/its/road/action_plan/traffic-information_en
% https://www.itwissen.info/RTTI-realtime-traffic-information.html
% https://ieeexplore.ieee.org/document/861224

The most up-to-date variation of a traffic status information system is the \acrfull{atis}, such as GoogleMaps, HERE or Waze. Like \acrshort{rtti} it is a provider-based service, but mostly without costs and less device constraint. Because of these missing accessibility constraints and the fact that in current times nearly 70\% of people carry a smartphone \parencite{IZM2020}, the user base of such services is quite substantial. This does not invalidate the usage of \acrshort{rtti} and \acrshort{tmc}, since they are present in most separate and built-in navigational systems, but rather makes \acrshort{atis} a considerable alternative. With the added benefit of being able to not only publish information about the traffic situation but also collect variations of \acrshort{fcd}, \acrshort{atis} is the most promising technology for \acrshort{atdm} and \acrshort{fcd} collection. 

\bigskip

In the scope of this thesis only the following  selection of these introduced data collection systems will be further relevant. 
\begin{itemize}
  \item \acrshort{fcd} for the congestion data
  \item \acrshort{baysis} and \acrshort{arbis} for incident data
  \item \acrshort{atis} for the concept of \acrshort{atdm}
\end{itemize}

%\textcite{} 
%\footcite{}
%\footnote{}.
% \begin{figure}[ht]
%     \includegraphics[width=\textwidth, height=\textheight,keepaspectratio]{}
%     \caption[]{} 
%     \label{fig:}
% \end{figure}
