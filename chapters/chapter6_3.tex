% ----------------------------------------------------
% -------- BAYSIS - Selected as Jam Effector ---------
% ----------------------------------------------------
\subsection{Congestion - Accidents categorizes as Jam Effector}
\label{analysis_processing_correlation_baysis_effector}
The correlation matrix table for the congestion - accident dataset, which are classified as \textit{Jam Effector} (see \cref{table:appendix_correlation_matrix_matched_cramers}) is visual presented in \cref{img:correlation_matrix_selected_effector_cramers} showing the the correlation of each variable combination. When visual analyzing \cref{img:correlation_matrix_matched_cramers} and checking the guidelines for a strong correlation in reference to the applied coefficient (identifiable with \cref{table:appendix_coefficient_matrix_matched}) we get a list of strongly correlated variable combinations (see \cref{tbl:correlation_list_baysis_effector}). Since the focus of the thesis are the correlations between accidents and jams, these are only collected from the bottom-left rectangle of the matrix, where the congestion and accidents variables intersect. Correlations of the kind congestion - congestion or accident - accident are not considered.
\begin{table}[h!]
	\centering
	\begin{tabular}{c|l}  
		Category & Strong \\
		\\[-1em]
		\hline
		\\[-1em]
		Strasse & TMax, TAvg, SMax, SAvg, Cov, TLHGV \\ 
 		Kat & TMax, TAvg, SAvg \\ % + SMax % -> Strasse
 		%Typ & \\ % -> Strasse
 		%Betei & \\
 		UArt1 & SAvg \\ % -> Strasse
 		%UArt2 & \\ % -> Strasse
 		%AUrs1 & \\ % -> Strasse
 		%AUrs2 & \\
 		AufHi & TMax, TAvg \\
 		%Alkoh & \\
 		%Char1 & \\ % -> Strasse
 		%Char2 & \\ % -> Strasse
 		%Lich1 & \\ % -> Strasse
 		%Lich2 & \\ % -> Strasse
 		%Zust1 & \\ % -> Strasse
 		%Zust2 & \\ % -> Strasse
 		%Fstf & \\ % -> Strasse
 		WoTag & TAvg, SMax, Cov, TLHGV \\ % -> Strasse
 		%FeiTag & \\
 		Month & TMax, TAvg, SMax, SAvg, Cov, TLHGV \\ % -> Strasse
	\end{tabular}
    \caption{List of incident variables and their strong correlated congestion variable from the congestion-accident matched data which are classified as \textit{Jam Effector}}
	\label{tbl:correlation_list_baysis_effector}
\end{table}
Next we need to verify that the correlation is significant and what the correlation predicates. Therefore each correlation will be evaluated with the Post Hoc test, defined in \cref{correlation_posthoc}. \secintroend{baysis}{effector}
\begin{figure}[!ht]
	\centering
	\makebox[\textwidth][c]{%
		\includegraphics[width=1.4\textwidth, trim=0cm 2.5cm 6cm 3cm]{code/data/BAYSIS/03_selected_02_duringJam/plots/baysis_selected_corr_cramers_edited}%
	}
	\caption{Correlation matrix for congestion-accident matched data classified as \textit{Jam Effector} and calculated with $V$, $\eta$, $\tau$, $r_{pq}$, $r$}
	\label{img:correlation_matrix_selected_effector_cramers}
\end{figure}

% --------------------------
% -------- Street ---------
% --------------------------
\centerheading{Street}
\label{ana:baysis_effector_Str}
\varintrosimplewithsam{Str}
\varintronosigmul{Str}{\textit{Strasse} - \textit{SAvg} and \textit{Strasse} - \textit{TLHGV}}

% #################################################
\groupintrosig{Street}{TMax}{0.0104}{baysis}{effector}
\begin{table}[ht!]
	\tiny
	\centering
	\begin{tabular}{rrrrrrrrrrrrrr}
		\toprule
		     & A3 & A6 & A9 & A70 & A99 & A93 & A94 & A7 & A73 & A96 & A995 & A92 & A95 \\ 
		\midrule
		% A6   & 1.00 &  &  &  &  &  &  &  &  &  &  &  &  \\ 
		% A9   & 1.00 & 1.00 &  &  &  &  &  &  &  &  &  &  &  \\ 
		% A70  & 1.00 & 1.00 & 1.00 &  &  &  &  &  &  &  &  &  &  \\ 
		% A99  & 0.85 & 1.00 & 1.00 & 1.00 &  &  &  &  &  &  &  &  &  \\ 
		% A93  & 1.00 & 1.00 & 1.00 & 1.00 & 1.00 &  &  &  &  &  &  &  &  \\ 
		A94  & \red{0.01} & 1.00 & 0.06 & 1.00 & 0.28 & 1.00 &  &  &  &  &  &  &  \\ 
		% A7   & .00 & 1.00 & 1.00 & 1.00 & 1.00 & 1.00 & 1.00 &  &  &  &  &  &  \\ 
		A73  & \red{0.00} & 1.00 & \red{0.00} & 1.00 & 0.23 & 1.00 & 1.00 & 1.00 &  &  &  &  &  \\ 
		A96  & \red{0.00} & 1.00 & 0.35 & 1.00 & 1.00 & 1.00 & 1.00 & 1.00 & 1.00 &  &  &  &  \\ 
		% A995 & 1.00 & 1.00 & 1.00 & 1.00 & 1.00 & 1.00 & 1.00 & 1.00 & 1.00 & 1.00 &  &  &  \\ 
		A92  & \red{0.01} & 1.00 & 0.24 & 1.00 & 1.00 & 1.00 & 1.00 & 1.00 & 1.00 & 1.00 & 1.00 &  &  \\ 
		% A95  & 1.00 & 1.00 & 1.00 & 1.00 & 1.00 & 1.00 & 1.00 & 1.00 & 1.00 & 1.00 & 1.00 & 1.00 &  \\ 
		% A980 & 1.00 & 1.00 & 1.00 & 1.00 & 1.00 & 1.00 & 1.00 & 1.00 & 1.00 & 1.00 & 1.00 & 1.00 &  \\ 
		\bottomrule
	  \end{tabular}
    \caption{Pairwise Wilcoxon $T$-test for \textit{Street} and \textit{Maximal Temporal Extent} (Jam Effector)}
    \label{tbl:wilcoxon_baysis_effector_Street_TMax}
\end{table}
The table shows that the roads A73, A94, A95 and A96 differ from the A3. The A73 also differs from the A9, but there is no distinctive general trend.
% #### START: Table and Plot
\begin{figure}[ht!]
	\centering
	\begin{minipage}{0.5\textwidth}
		\tiny
		\setlength{\tabcolsep}{4pt}
		\centering
		\begin{tabular}{c|c|c|c|c|c|c|c}
			\toprule
			Group & $n$ & $\bar{x}$ & $\sigma$ & $\tilde{x}$ & $min$ & $max$ & $\Delta$ \\
			\midrule
			A3   & 265 & 297.62 & 219.09 & 243.0 & 18 & 1257 & 1239 \\ 
			A6   & 37  & 236.59 & 183.03 & 207.0 & 18 & 705  & 687  \\ 
			A9   & 192 & 243.45 & 178.97 & 201.0 & 15 & 1194 & 1179 \\ 
			A99  & 63  & 212.57 & 140.78 & 195.0 & 21 & 681  & 660  \\ 
			A93  & 12  & 212.00 & 187.74 & 130.5 & 39 & 588  & 549  \\ 
			A94  & 14  & 109.71 & 56.07  & 97.5  & 42 & 249  & 207  \\ 
			A7   & 35  & 254.66 & 306.21 & 168.0 & 42 & 1341 & 1299 \\ 
			A73  & 52  & 157.21 & 179.42 & 130.5 & 18 & 1323 & 1305 \\ 
			A96  & 41  & 155.85 & 84.33  & 144.0 & 30 & 381  & 351  \\ 
			A92  & 21  & 134.71 & 82.25  & 138.0 & 18 & 354  & 336  \\ 
			\bottomrule
			% \bar{x} - sum = 2014.37, mean = 201,43
			% \sigma - sum = 1617.89, mean = 161.78
			% \tilde{x}
		\end{tabular}
		\subcaption[second caption.]{Table of all descriptives}\label{tbl:descriptives_baysis_effector_Street_TMax}
	\end{minipage}%
	\begin{minipage}{0.55\textwidth}
		\pgfplotstableread[col sep=comma]{
			road, count, mean, median, sd, min, max, delta
			A3   , 265 , 297.62 , 219.09 , 243.0 , 18 , 1257 , 1239
			A6   , 37  , 236.59 , 183.03 , 207.0 , 18 , 705  , 687 
			A9   , 192 , 243.45 , 178.97 , 201.0 , 15 , 1194 , 1179
			A99  , 63  , 212.57 , 140.78 , 195.0 , 21 , 681  , 660 
			A93  , 12  , 212.00 , 187.74 , 130.5 , 39 , 588  , 549 
			A94  , 14  , 109.71 , 56.07  , 97.5  , 42 , 249  , 207 
			A7   , 35  , 254.66 , 306.21 , 168.0 , 42 , 1341 , 1299
			A73  , 52  , 157.21 , 179.42 , 130.5 , 18 , 1323 , 1305
			A96  , 41  , 155.85 , 84.33  , 144.0 , 30 , 381  , 351 
			A92  , 21  , 134.71 , 82.25  , 138.0 , 18 , 354  , 336 
		}\data
        \pgfplotstablesort[sort key=mean, sort cmp=float >]{\datasorted}{\data}
        \tiny
        \centering
        \descplotfigwithavg{\datasorted}{201}{161}{9}{4.7}
		\subcaption[second caption.]{Plot of descriptives $\bar{x}$, $\sigma$ and $\tilde{x}$}\label{fig:descriptives_baysis_effector_Street_TMax}
	\end{minipage}%
	\caption{Group descriptives of \textit{Street} and \textit{Maximal Temporal Extent} (Jam Effector)}
	%\vspace{-8mm}
\end{figure}
% #### END: Table and Plot
The significant descriptives from \cref{tbl:descriptives_baysis_effector_Street_TMax,fig:descriptives_baysis_effector_Street_TMax} that the mean of A3 is 140\,min - 188\,min higher than the means of A73, A94, A95 and A96. They also show that the mean of A73 is 86\,min lower than the mean of A9. Therefore it can be interpreted that accidents on the A3 and A9 are associated with significantly longer (temporal) jams than on the A73, A94, A95 and A96. The descriptives also show that the A73, A92, A94 and A96 are have considerable shorter durations, when the A3, A6, A7 and A9 have considerable longer durations compared to the general mean.
\groupintrosig{Street}{TAvg}{0.0003}{baysis}{effector}
\begin{table}[ht!]
	\tiny
	\centering
	\begin{tabular}{rrrrrrrrrrrrrr}
		\toprule
			 & A3 & A6 & A9 & A70 & A99 & A93 & A94 & A7 & A73 & A96 & A995 & A92 & A95 \\ 
		\midrule
		% A6   & 1.00 &  &  &  &  &  &  &  &  &  &  &  &  \\ 
		% A9   & 1.00 & 1.00 &  &  &  &  &  &  &  &  &  &  &  \\ 
		% A70  & 1.00 & 1.00 & 1.00 &  &  &  &  &  &  &  &  &  &  \\ 
		A99  & \red{0.02} & 1.00 & 1.00 & 1.00 &  &  &  &  &  &  &  &  &  \\ 
		% A93  & 1.00 & 1.00 & 1.00 & 1.00 & 1.00 &  &  &  &  &  &  &  &  \\ 
		% A94  & 0.11 & 1.00 & 0.53 & 1.00 & 1.00 & 1.00 &  &  &  &  &  &  &  \\ 
		% A7   & 1.00 & 1.00 & 1.00 & 1.00 & 1.00 & 1.00 & 0.61 &  &  &  &  &  &  \\ 
		A73  & \red{0.00} & 1.00 & 0.00 & 1.00 & 1.00 & 1.00 & 1.00 & \red{0.02} &  &  &  &  &  \\ 
		% A96  & 1.00 & 1.00 & 1.00 & 1.00 & 1.00 & 1.00 & 1.00 & 1.00 & 0.87 &  &  &  &  \\ 
		% A995 & 1.00 & 1.00 & 1.00 & 1.00 & 1.00 & 1.00 & 1.00 & 1.00 & 1.00 & 1.00 &  &  &  \\ 
		% A92  & 1.00 & 1.00 & 1.00 & 1.00 & 1.00 & 1.00 & 1.00 & 1.00 & 1.00 & 1.00 & 1.00 &  &  \\ 
		% A95  & 1.00 & 1.00 & 1.00 & 1.00 & 1.00 & 1.00 & 1.00 & 1.00 & 1.00 & 1.00 & 1.00 & 1.00 &  \\ 
		% A980 & 1.00 & 1.00 & 1.00 & 1.00 & 1.00 & 1.00 & 1.00 & 1.00 & 1.00 & 1.00 & 1.00 & 1.00 & 1.00 \\ 
		\bottomrule
	  \end{tabular}
    \caption{Pairwise Wilcoxon $T$-test for \textit{Strasse} and \textit{Average Temporal Extent} (Jam Effector)}
    \label{tbl:wilcoxon_baysis_effector_Street_TAvg}
\end{table}
The table shows that the roads A99 and A73 differ significantly from the A3. The A73 also differs significantly from A7.
% #### START: Table and Plot
\begin{figure}[ht!]
	\centering
	\begin{minipage}{0.5\textwidth}
		\tiny
		\centering
		\begin{tabular}{c|c|c|c|c|c|c|c}
			\toprule
			Group & $n$ & $\bar{x}$ & $\sigma$ & $\tilde{x}$ & $min$ & $max$ & $\Delta$ \\
			\midrule
			A3   & 265 & 104.05 & 87.02  & 84 & 7  & 703  & 696  \\ 
			A6   & 37  & 82.43  & 74.19  & 70 & 3  & 301  & 298  \\ 
			A9   & 192 & 91.31  & 74.64  & 75 & 5  & 575  & 570  \\  
			A99  & 63  & 65.73  & 47.32  & 52 & 4  & 295  & 291  \\ 
			A93  & 12  & 104.83 & 112.03 & 55 & 7  & 343  & 336  \\ 
			A94  & 14  & 45.93  & 25.54  & 43 & 14 & 102  & 88   \\ 
			A7   & 35  & 143.74 & 255.78 & 76 & 15 & 1326 & 1311 \\ 
			A73  & 52  & 47.63  & 29.09  & 44 & 6  & 154  & 148  \\ 
			A96  & 41  & 74.39  & 54.53  & 70 & 6  & 247  & 241  \\ 
			A92  & 21  & 64.52  & 48.86  & 56 & 8  & 235  & 227  \\ 
			\bottomrule
			% \bar{x} - sum = 824.56, mean = 82.45
			% \sigma - sum = 809, mean = 80.9
			% \tilde{x}
		\end{tabular}
		\subcaption[second caption.]{Table of all descriptives}\label{tbl:descriptives_baysis_effector_Street_TAvg}
	\end{minipage}%
	\begin{minipage}{0.55\textwidth}
		\pgfplotstableread[col sep=comma]{
			road, count, mean, median, sd, min, max, delta
			A3   , 265 , 104.05 , 87.02  , 84 , 7  , 703  , 696 
			A6   , 37  , 82.43  , 74.19  , 70 , 3  , 301  , 298 
			A9   , 192 , 91.31  , 74.64  , 75 , 5  , 575  , 570  
			A99  , 63  , 65.73  , 47.32  , 52 , 4  , 295  , 291 
			A93  , 12  , 104.83 , 112.03 , 55 , 7  , 343  , 336 
			A94  , 14  , 45.93  , 25.54  , 43 , 14 , 102  , 88  
			A7   , 35  , 143.74 , 255.78 , 76 , 15 , 1326 , 1311
			A73  , 52  , 47.63  , 29.09  , 44 , 6  , 154  , 148 
			A96  , 41  , 74.39  , 54.53  , 70 , 6  , 247  , 241 
			A92  , 21  , 64.52  , 48.86  , 56 , 8  , 235  , 227 
		}\data
		\pgfplotstablesort[sort key=mean, sort cmp=float >]{\datasorted}{\data}
        \tiny
        \centering
        \descplotfigwithavg{\datasorted}{82}{80}{9}{4.7}
		\subcaption[second caption.]{Plot of descriptives $\bar{x}$, $\sigma$ and $\tilde{x}$}\label{fig:descriptives_baysis_effector_Street_TAvg}
	\end{minipage}%
	\caption{Group descriptives of \textit{Street} and \textit{Average Temporal Extent} (Jam Effector)}
	%\vspace{-8mm}
\end{figure}
% #### END: Table and Plot
The significant descriptives from \cref{tbl:descriptives_baysis_effector_Street_TAvg,fig:descriptives_baysis_effector_Street_TAvg} that the mean of A3 is 39\,min - 57\,min higher than the means of A99 and A73. They also show that the mean of A73 is 100\,min lower than the mean of A7, which breaks the general trend of the variable and could be the result of errors. Never the less it can be interpreted that accidents on the A3 and A7 are associated with significantly longer (temporal) jams than on the A99 and A73. The descriptives also show that the A73, A92, A94 and A96 are have considerable shorter durations, when the A3, A7, A9 and A99 have considerable longer durations compared to the general mean.
\begin{figure}[ht!]
	\pgfplotstableread[col sep=comma]{
		Road, meanTMax, meanTAvg
		A3  , 297.62  , 104.05
		A6  , 236.59  , 82.43 
		A9  , 243.45  , 91.31 
		A99 , 212.57  , 65.73 
		A93 , 212.00  , 104.83
		A94 , 109.71  , 45.93 
		A7  , 254.66  , 143.74
		A73 , 157.21  , 47.63 
		A96 , 155.85  , 74.39 
		A92 , 134.71  , 64.52   
	}\data 
	\pgfplotstablesort[sort key=meanTAvg, sort cmp=float >]{\datasorted}{\data}
	\tiny
	\centering
	\barplotdouble{\datasorted}{meanTMax}{meanTAvg}{$\bar{x}_{TMax}$}{$\bar{x}_{TAvg}$}
	\caption{Comparison of descriptives $\bar{x}_{TMax}$ and $\bar{x}_{TAvg}$ (\textit{TMax/TAvg} by \textit{Street}) (Jam Effector)}
	\label{fig:baysis_effector_meancomparison_Str_temporal}
	%\vspace{-8mm}
\end{figure}
When comparing the mean values of the maximal and average (temporal) extend (shown in \cref{fig:baysis_effector_meancomparison_Str_temporal}) it becomes clear that the average variable has considerable lower values than the maximum variable, which is to be expected. It also shows, that the differences between the groups are mostly different in the maximal and average extend and vary considerably. In can be described that they follow tend, but A3, A9, A6 and A99 have higher maximal durations than the average trend.

% ####################################################
\groupintrosig{Street}{SMax}{0.0025}{baysis}{effector}
\begin{table}[ht!]
	\tiny
	\centering
	\begin{tabular}{rrrrrrrrrrrrrr}
		\toprule
			 & A3 & A6 & A9 & A70 & A99 & A93 & A94 & A7 & A73 & A96 & A995 & A92 & A95 \\ 
		\midrule
		% A6   & 1.00 &  &  &  &  &  &  &  &  &  &  &  &  \\ 
		A9   & \red{0.00} & 1.00 &  &  &  &  &  &  &  &  &  &  &  \\ 
		% A70  & 1.00 & 1.00 & 1.00 &  &  &  &  &  &  &  &  &  &  \\ 
		% A99  & 1.00 & 1.00 & 1.00 & 1.00 &  &  &  &  &  &  &  &  &  \\ 
		A93  & 0.07 & 0.13 & 1.00 & 1.00 & 1.00 &  &  &  &  &  &  &  &  \\ 
		A94  & \red{0.01} & \red{0.03} & 0.41 & 1.00 & 0.24 & 1.00 &  &  &  &  &  &  &  \\ 
		A7   & 0.08 & 0.65 & 1.00 & 1.00 & 1.00 & 1.00 & 1.00 &  &  &  &  &  &  \\ 
		A73  & \red{0.00} & \red{0.00} & \red{0.00} & 1.00 & \red{0.00} & 1.00 & 1.00 & 0.95 &  &  &  &  &  \\ 
		A96  & 0.13 & 1.00 & 1.00 & 1.00 & 1.00 & 1.00 & 1.00 & 1.00 & 0.18 &  &  &  &  \\ 
		% A995 & 1.00 & 1.00 & 1.00 & 1.00 & 1.00 & 1.00 & 1.00 & 1.00 & 1.00 & 1.00 &  &  &  \\ 
		A92  & \red{0.00} & \red{0.00} & \red{0.04} & 1.00 & \red{0.04} & 1.00 & 1.00 & 1.00 & 1.00 & 1.00 & 1.00 &  &  \\ 
		% A95  & 1.00 & 1.00 & 1.00 & 1.00 & 1.00 & 1.00 & 1.00 & 1.00 & 1.00 & 1.00 & 1.00 & 1.00 &  \\ 
		% A980 & 1.00 & 1.00 & 1.00 & 1.00 & 1.00 & 1.00 & 1.00 & 1.00 & 1.00 & 1.00 & 1.00 & 1.00 & 1.00 \\ 
		\bottomrule
	  \end{tabular}
    \caption{Pairwise Wilcoxon $T$-test for \textit{Street} and \textit{Maximal Spatial Extent} (Jam Effector)}
    \label{tbl:wilcoxon_baysis_effector_Street_SMax}
\end{table}
The table shows that the roads of A73, A9, A92 and A94 differ significantly from A3. The roads A73, A92 and A93 also differ significantly from A6. The A73 and A92 also differ significantly from A9 and A99, but there is no distinctive trend.
% #### START: Table and Plot
\begin{figure}[ht!]
	\centering
	\begin{minipage}{0.5\textwidth}
		\tiny
		\setlength{\tabcolsep}{4pt}
		\centering
		\begin{tabular}{c|c|c|c|c|c|c|c}
			\toprule
			Group & $n$ & $\bar{x}$ & $\sigma$ & $\tilde{x}$ & $min$ & $max$ & $\Delta$ \\
			\midrule
			A3   & 265 & 17755.15 & 11139.22 & 14811.0 & 1566 & 46328 & 44762 \\ 
			A6   & 37  & 16711.43 & 9518.81  & 14449.0 & 2655 & 40033 & 37378 \\ 
			A9   & 192 & 13271.42 & 8473.71  & 11654.5 & 1315 & 49765 & 48450 \\ 
			A99  & 63  & 17558.70 & 12223.42 & 14698.0 & 2351 & 48278 & 45927 \\ 
			A93  & 12  & 8158.42  & 4473.23  & 7461.5  & 2896 & 16922 & 14026 \\ 
			A94  & 14  & 7277.64  & 4218.45  & 6947.0  & 1206 & 15550 & 14344 \\ 
			A7   & 35  & 11825.00 & 8885.99  & 9506.0  & 3041 & 43244 & 40203 \\ 
			A73  & 52  & 7964.98  & 5995.59  & 6559.5  & 1036 & 33764 & 32728 \\ 
			A96  & 41  & 11825.98 & 6911.19  & 9676.0  & 2006 & 27965 & 25959 \\ 
			A92  & 21  & 7255.76  & 3637.71  & 7364.0  & 1176 & 13522 & 12346 \\ 
			\bottomrule
			% \bar{x} - sum = 119604.48, mean = 11960.44
			% \sigma - sum = 75477.32, mean = 7547.73
			% \tilde{x}
		\end{tabular}
		\subcaption[second caption.]{Table of all descriptives}\label{tbl:descriptives_baysis_effector_Street_SMax}
	\end{minipage}%
	\begin{minipage}{0.55\textwidth}
		\pgfplotstableread[col sep=comma]{
			road, count, mean, median, sd, min, max, delta
			A3   , 265 , 17755.15 , 11139.22 , 14811.0 , 1566 , 46328 , 44762 
			A6   , 37  , 16711.43 , 9518.81  , 14449.0 , 2655 , 40033 , 37378 
			A9   , 192 , 13271.42 , 8473.71  , 11654.5 , 1315 , 49765 , 48450 
			A99  , 63  , 17558.70 , 12223.42 , 14698.0 , 2351 , 48278 , 45927 
			A93  , 12  , 8158.42  , 4473.23  , 7461.5  , 2896 , 16922 , 14026 
			A94  , 14  , 7277.64  , 4218.45  , 6947.0  , 1206 , 15550 , 14344 
			A7   , 35  , 11825.00 , 8885.99  , 9506.0  , 3041 , 43244 , 40203 
			A73  , 52  , 7964.98  , 5995.59  , 6559.5  , 1036 , 33764 , 32728 
			A96  , 41  , 11825.98 , 6911.19  , 9676.0  , 2006 , 27965 , 25959 
			A92  , 21  , 7255.76  , 3637.71  , 7364.0  , 1176 , 13522 , 12346 
		}\data
		\pgfplotstablesort[sort key=mean, sort cmp=float >]{\datasorted}{\data}
		\tiny
		\centering
		\descplotfigwithcustomavg{\datasorted}{11960}{7547}{1.19}{0.75}{9}{4.7}
		\subcaption[second caption.]{Plot of descriptives $\bar{x}$, $\sigma$ and $\tilde{x}$}\label{fig:descriptives_baysis_effector_Street_SMax}
	\end{minipage}%
	\caption{Group descriptives of \textit{Street} and \textit{Maximal Spatial Extent} (Jam Effector)}
	%\vspace{-8mm}
\end{figure}
% #### END: Table and Plot
The significant descriptives from \cref{tbl:descriptives_baysis_effector_Street_SMax,fig:descriptives_baysis_effector_Street_SMax} show that the mean of A3 is 4484\,m - 10500\,m higher than the means of A9, A73, A92 and A94. They also shows that the groups A6 have a 9212\,m higher mean on average than the groups A73 and A92. The mean of the groups A9 and A99 are 7805\,m higher than the A73 and A92. Therefore it can be interpreted that accidents on the A3, A6, A9 and A99 are associated with significantly longer (spatial) jams than on A9, A73, A92 and A94. The descriptives show also that the A3, A6, A9 and A99 have a considerable longer lengths, when the A73, A92, A93 and A94 have considerable shorter lengths compared to the general mean.

% ####################################################
\groupintrosig{Street}{Cov}{0.0055}{baysis}{effector}
\begin{table}[ht!]
	\tiny
	\centering
	\begin{tabular}{rrrrrrrrrrrrrr}
		\toprule
			 & A3 & A6 & A9 & A70 & A99 & A93 & A94 & A7 & A73 & A96 & A995 & A92 & A95 \\ 
		\midrule
		% A6   & 1.00 &  &  &  &  &  &  &  &  &  &  &  &  \\ 
		A9   & \red{0.01} & 1.00 &  &  &  &  &  &  &  &  &  &  &  \\ 
		% A70  & 1.00 & 1.00 & 1.00 &  &  &  &  &  &  &  &  &  &  \\ 
		A99  & 1.00 & 1.00 & \red{0.00} & 1.00 &  &  &  &  &  &  &  &  &  \\ 
		% A93  & 1.00 & 1.00 & 1.00 & 1.00 & 1.00 &  &  &  &  &  &  &  &  \\ 
		% A94  & 1.00 & 1.00 & 1.00 & 1.00 & 1.00 & 1.00 &  &  &  &  &  &  &  \\ 
		A7   & 0.25 & 1.00 & 1.00 & 1.00 & \red{0.02} & 1.00 & 1.00 &  &  &  &  &  &  \\ 
		% A73  & 1.00 & 1.00 & 1.00 & 1.00 & 1.00 & 1.00 & 1.00 & 1.00 &  &  &  &  &  \\ 
		A96  & 0.21 & 1.00 & 1.00 & 1.00 & \red{0.02} & 1.00 & 1.00 & 1.00 & 1.00 &  &  &  &  \\ 
		% A995 & 1.00 & 1.00 & 1.00 & 1.00 & 1.00 & 1.00 & 1.00 & 1.00 & 1.00 & 1.00 &  &  &  \\ 
		A92  & \red{0.03} & 0.43 & 1.00 & 1.00 & \red{0.01} & 1.00 & 1.00 & 1.00 & 0.61 & 1.00 & 1.00 &  &  \\ 
		% A95  & 1.00 & 1.00 & 1.00 & 1.00 & 1.00 & 1.00 & 1.00 & 1.00 & 1.00 & 1.00 & 1.00 & 1.00 &  \\ 
		% A980 & 1.00 & 1.00 & 1.00 & 1.00 & 1.00 & 1.00 & 1.00 & 1.00 & 1.00 & 1.00 & 1.00 & 1.00 & 1.00 \\ 
		\bottomrule
	  \end{tabular}
    \caption{Pairwise Wilcoxon $T$-test for \textit{Street} and \textit{Coverage} (Jam Effector)}
    \label{tbl:wilcoxon_baysis_effector_Street_Cov}
\end{table}
The table shows that roads A9 and A92 differ significantly from A3. The road A99 differs significantly A9. The roads A7, A92 and A96 differ significantly from A99.
% #### START: Table and Plot
\begin{figure}[ht!]
	\centering
	\begin{minipage}{0.5\textwidth}
		\tiny
		\setlength{\tabcolsep}{4pt}
		\centering
		\begin{tabular}{c|c|c|c|c|c|c|c}
			\toprule
			Group & $n$ & $\bar{x}$ & $\sigma$ & $\tilde{x}$ & $min$ & $max$ & $\Delta$ \\
			\midrule
			A3   & 265 & 32.02 & 16.40 & 28.0 & 2  & 100 & 92 \\ 
			A6   & 37  & 30.81 & 16.61 & 25.0 & 9  & 65  & 56 \\ 
			A9   & 192 & 36.09 & 13.77 & 34.5 & 6  & 86  & 80 \\ 
			A99  & 63  & 26.79 & 14.71 & 25.0 & 7  & 63  & 56 \\ 
			A93  & 12  & 37.25 & 17.78 & 36.5 & 13 & 70  & 57 \\ 
			A94  & 14  & 36.79 & 21.00 & 33.5 & 11 & 77  & 66 \\ 
			A7   & 35  & 45.60 & 25.54 & 42.0 & 6  & 100 & 94 \\ 
			A73  & 52  & 32.94 & 15.73 & 30.5 & 7  & 77  & 70 \\ 
			A96  & 41  & 42.90 & 21.63 & 42.0 & 9  & 85  & 76 \\ 
			A92  & 21  & 47.71 & 22.04 & 47.0 & 21 & 88  & 67 \\ 
			\bottomrule
			% \bar{x} - sum = 368.9, mean = 36.89
			% \sigma - sum = 185.21, mean = 18.52
			% \tilde{x}
		\end{tabular}
		\subcaption[second caption.]{Table of all descriptives}\label{tbl:descriptives_baysis_effector_Street_Cov}
	\end{minipage}%
	\begin{minipage}{0.55\textwidth}
		\pgfplotstableread[col sep=comma]{
			road, count, mean, median, sd, min, max, delta
			A3   , 265 , 32.02 , 16.40 , 28.0 , 2  , 100 , 92 
			A6   , 37  , 30.81 , 16.61 , 25.0 , 9  , 65  , 56 
			A9   , 192 , 36.09 , 13.77 , 34.5 , 6  , 86  , 80 
			A99  , 63  , 26.79 , 14.71 , 25.0 , 7  , 63  , 56 
			A93  , 12  , 37.25 , 17.78 , 36.5 , 13 , 70  , 57 
			A94  , 14  , 36.79 , 21.00 , 33.5 , 11 , 77  , 66 
			A7   , 35  , 45.60 , 25.54 , 42.0 , 6  , 100 , 94 
			A73  , 52  , 32.94 , 15.73 , 30.5 , 7  , 77  , 70 
			A96  , 41  , 42.90 , 21.63 , 42.0 , 9  , 85  , 76 
			A92  , 21  , 47.71 , 22.04 , 47.0 , 21 , 88  , 67 
		}\data
		\pgfplotstablesort[sort key=mean, sort cmp=float >]{\datasorted}{\data}
		\tiny
		\centering
		\descplotfigwithavg{\datasorted}{36}{18}{9}{4.7}
		\subcaption[second caption.]{Plot of descriptives $\bar{x}$, $\sigma$ and $\tilde{x}$}\label{fig:descriptives_baysis_effector_Street_Cov}
	\end{minipage}%
	\caption{Group descriptives of \textit{Street} and \textit{Coverage} (Jam Effector)}
	%\vspace{-8mm}
\end{figure}
% #### END: Table and Plot
The significant descriptives from \cref{tbl:descriptives_baysis_effector_Street_Cov,fig:descriptives_baysis_effector_Street_Cov} show that the mean of A3 is +4\,\% and -6\,\% different to the means of A9 and A92 respectively. They also shows that the mean of A9 is about 10\,\% higher than the mean of A99, which has a 16\,\% - 21\,\% lower mean on average than the A7, A92 and A96. Therefore it can be interpreted that accidents on the A3 and A99 are associated with significantly less dense jams than on A7, A9, A92 and A96. The descriptives show also that the A7, A92 and A96 are have considerable higher coverage, when the A3, A6, A73 and A99 have considerable lower coverage compared to the general mean.

% ----------------------
% -------- Kat ---------
% ----------------------
\centerheading{Kat}
\varintronosigmul{Kat}{\textit{Kat} - \textit{TMax}, \textit{Kat} - \textit{TAvg} and \textit{Kat} - \textit{SAvg}}

% ----------------------
% -------- Typ ---------
% ----------------------

% -----------------------
% -------- UArt ---------
% -----------------------
\centerheading{UArt}
\varintronosigsing{UArt1}{SAvg}

% -----------------------
% -------- AUrs ---------
% -----------------------

% ------------------------
% -------- AufHi ---------
% ------------------------
\centerheading{AufHi}
\varintronosigdouble{AufHi}{\textit{AufHi} - \textit{TMax} and \textit{AufHi} - \textit{TAvg}}

% -----------------------
% -------- Char ---------
% -----------------------

% -----------------------
% -------- Lich ---------
% -----------------------

% ------------------------
% -------- WoTag ---------
% ------------------------
\centerheading{WoTag}
\varintrosimplewithsam{WoTag} \varintronosigdouble{WoTag}{\textit{WoTag} - \textit{TAvg} and \textit{WoTag} - \textit{Cov}}

% ####################################################
\groupintrosig{WoTag}{SMax}{0.0175}{baysis}{effector} 
Although the Kruskal-Wallis test shows significant differences, the Wilcoxon table shows that the differences are not group specific.
% #### START: Table and Plot
\begin{figure}[ht!]
	\centering
	\begin{minipage}{0.5\textwidth}
		\tiny
		\setlength{\tabcolsep}{4pt}
		\centering
		\begin{tabular}{c|c|c|c|c|c|c|c}
			\toprule
			Group & $n$ & $\bar{x}$ & $\sigma$ & $\tilde{x}$ & $min$ & $max$ & $\Delta$ \\
			\midrule
			Mo & 115 & 14126.74 & 8746.49  & 12433 & 1315 & 38867 & 37552 \\
			Di & 129 & 14769.57 & 9871.55  & 12827 & 2006 & 42736 & 40730 \\
			Mi & 142 & 14790.29 & 11226.25 & 12345 & 1176 & 48278 & 47102 \\
			Do & 124 & 15481.44 & 10576.67 & 13163 & 1772 & 42658 & 40886 \\
			Fr & 89  & 12327.72 & 8406.29  & 10022 & 1036 & 49765 & 48729 \\
			Sa & 65  & 14951.55 & 11527.80 & 10986 & 2402 & 46328 & 43926 \\
			So & 73  & 13498.93 & 9284.69  & 10825 & 1524 & 42393 & 40869 \\
			\bottomrule
			% \bar{x} - sum = 99946.24, mean = 14278.03
			% \sigma - sum = 69639.72, mean = 9948.53
			% \tilde{x}
		\end{tabular}
		\subcaption[second caption.]{Table of all descriptives}\label{tbl:descriptives_baysis_effector_WoTag_SMax}
	\end{minipage}%
	\begin{minipage}{0.55\textwidth}
		\pgfplotstableread[col sep=comma]{
			day, count, mean, median, sd, min, max, delta
			Mo , 115 , 14126.74 , 8746.49  , 12433 , 1315 , 38867 , 37552
			Di , 129 , 14769.57 , 9871.55  , 12827 , 2006 , 42736 , 40730
			Mi , 142 , 14790.29 , 11226.25 , 12345 , 1176 , 48278 , 47102
			Do , 124 , 15481.44 , 10576.67 , 13163 , 1772 , 42658 , 40886
			Fr , 89  , 12327.72 , 8406.29  , 10022 , 1036 , 49765 , 48729
			Sa , 65  , 14951.55 , 11527.80 , 10986 , 2402 , 46328 , 43926
			So , 73  , 13498.93 , 9284.69  , 10825 , 1524 , 42393 , 40869 
		}\data
		\pgfplotstablesort[sort key=mean, sort cmp=float >]{\datasorted}{\data}
		\tiny
		\centering
		\descplotfigwithavg{\datasorted}{14278}{9948}{6}{4.7}
		\subcaption[second caption.]{Plot of descriptives $\bar{x}$, $\sigma$ and $\tilde{x}$}\label{fig:descriptives_baysis_effector_WoTag_SMax}
	\end{minipage}%
	\caption{Group descriptives of \textit{WoTag} and \textit{Time-loss HGV} (Jam Effector)}
	%\vspace{-8mm}
\end{figure}
% #### END: Table and Plot
The descriptives in \cref{tbl:descriptives_baysis_effector_WoTag_SMax,fig:descriptives_baysis_effector_WoTag_SMax} can still be interpreted. They show that accidents on Friday and Sunday can be associated with short jams, than on other weekdays.

% ####################################################
\groupintrosig{WoTag}{TLHGV}{0.0089}{baysis}{effector}
\begin{table}[ht!]
	\tiny
	\centering
    \begin{tabular}{rrrrrrr}
		\toprule
		   & Di & Do & Fr & Mi & Mo & Sa \\ 
		\midrule
		% Do & 1.00 &  &  &  &  &  \\ 
		% Fr & 1.00 & 1.00 &  &  &  &  \\ 
		Mi & \red{0.03} & 1.00 & 1.00 &  &  &  \\ 
		% Mo & 0.82 & 1.00 & 1.00 & 1.00 &  &  \\ 
		% Sa & 1.00 & 1.00 & 1.00 & 0.82 & 1.00 &  \\ 
		So & 1.00 & 1.00 & 1.00 & \red{0.05} & 0.65 & 1.00 \\ 
		\bottomrule
	\end{tabular}
    \caption{Pairwise Wilcoxon $T$-test for \textit{WoTag} and \textit{Time-loss HGV} (Jam Effector), see \cref{tbl:wilcoxon_baysis_effector_WoTag_TLHGV_complete} for complete table}
    \label{tbl:wilcoxon_baysis_effector_WoTag_TLHGV}
\end{table}
The table show, that the groups of Wednesday and Sunday differ significantly from Tuesday and Wednesday respectively.
% #### START: Table and Plot
\begin{figure}[ht!]
	\centering
	\begin{minipage}{0.5\textwidth}
		\tiny
		\setlength{\tabcolsep}{4pt}
		\centering
		\begin{tabular}{c|c|c|c|c|c|c|c}
			\toprule
			Group & $n$ & $\bar{x}$ & $\sigma$ & $\tilde{x}$ & $min$ & $max$ & $\Delta$ \\
			\midrule
			Mo & 89  & 728.10 & 134.08 & 727.00 & 514 & 986 & 472 \\ 
			Di & 115 & 766.97 & 141.79 & 794.00 & 518 & 999 & 481 \\ 
			Mi & 124 & 709.44 & 140.31 & 692.50 & 501 & 995 & 494 \\ 
			Do & 129 & 739.48 & 143.28 & 720.00 & 511 & 983 & 472 \\ 
			Fr & 142 & 739.78 & 155.44 & 720.00 & 502 & 998 & 496 \\ 
			Sa & 65  & 750.46 & 143.80 & 778.00 & 507 & 988 & 481 \\ 
			So & 73  & 774.67 & 141.34 & 784.00 & 534 & 994 & 460 \\ 
			\bottomrule
			% \bar{x} - sum = 5208.9, mean = 744.12
			% \sigma - sum = 1000.04, mean = 142.86
			% \tilde{x}
		\end{tabular}
		\subcaption[second caption.]{Table of all descriptives}\label{tbl:descriptives_baysis_effector_WoTag_TLHGV}
	\end{minipage}%
	\begin{minipage}{0.55\textwidth}
		\pgfplotstableread[col sep=comma]{
			day, count, mean, median, sd, min, max, delta
			Mo , 89  , 728.10 , 134.08 , 727.00 , 514 , 986 , 472 
			Di , 115 , 766.97 , 141.79 , 794.00 , 518 , 999 , 481 	
			Mi , 124 , 709.44 , 140.31 , 692.50 , 501 , 995 , 494 
			Do , 129 , 739.48 , 143.28 , 720.00 , 511 , 983 , 472 
			Fr , 142 , 739.78 , 155.44 , 720.00 , 502 , 998 , 496 
			Sa , 65  , 750.46 , 143.80 , 778.00 , 507 , 988 , 481 
			So , 73  , 774.67 , 141.34 , 784.00 , 534 , 994 , 460 
		}\data
		\pgfplotstablesort[sort key=mean, sort cmp=float >]{\datasorted}{\data}
		\tiny
		\centering
		\descplotfigwithavg{\datasorted}{744}{142}{6}{4.7}
		\subcaption[second caption.]{Plot of descriptives $\bar{x}$, $\sigma$ and $\tilde{x}$}\label{fig:descriptives_baysis_effector_WoTag_TLHGV}
	\end{minipage}%
	\caption{Group descriptives of \textit{WoTag} and \textit{Time-loss HGV} (Jam Effector)}
	%\vspace{-8mm}
\end{figure}
% #### END: Table and Plot
With the descriptives in \cref{tbl:descriptives_baysis_effector_WoTag_TLHGV,fig:descriptives_baysis_effector_WoTag_TLHGV} it can be interpreted, that the time loss for heavy goods vehicles on Tuesdays and Sundays (only theoretical, since there is no HGV traffic allow in Sundays) is 70 hours higher that on Wednesdays.

% ------------------------
% -------- Month ---------
% ------------------------
\centerheading{Month}
\varintrosimplewithsam{Month} \varintronosigmul{Month}{\textit{Month} - \textit{TMax}, \textit{Month} - \textit{TAvg}, \textit{Month} - \textit{TLHGV} and \textit{Month} - \textit{Cov}}

% ####################################################
\groupintrosig{Month}{SMax}{0.0208}{baysis}{effector}
\begin{table}[ht!]
	\tiny
	\centering
	\begin{tabular}{rrrrrrrrrrrr}
		\toprule
		    & Jan & Feb & Mar & Apr & May & Jun & Jul & Aug & Sep & Oct & Nov \\ 
		\midrule
		% Feb & 1.00 &  &  &  &  &  &  &  &  &  &  \\ 
		% Mar & 1.00 & 1.00 &  &  &  &  &  &  &  &  &  \\ 
		% Apr & 1.00 & 1.00 & 1.00 &  &  &  &  &  &  &  &  \\ 
		% May & 1.00 & 1.00 & 1.00 & 1.00 &  &  &  &  &  &  &  \\ 
		% Jun & 1.00 & 1.00 & 1.00 & 1.00 & 1.00 &  &  &  &  &  &  \\ 
		% Jul & 1.00 & 1.00 & 0.53 & 1.00 & 1.00 & 1.00 &  &  &  &  &  \\ 
		% Aug & 1.00 & 1.00 & 1.00 & 1.00 & 1.00 & 1.00 & 1.00 &  &  &  &  \\ 
		% Sep & 1.00 & 1.00 & 0.82 & 1.00 & 1.00 & 1.00 & 1.00 & 1.00 &  &  &  \\ 
		% Oct & 1.00 & 1.00 & 1.00 & 1.00 & 1.00 & 1.00 & 0.17 & 1.00 & 0.31 &  &  \\ 
		Nov & 1.00 & 1.00 & 1.00 & 1.00 & 1.00 & 1.00 & 0.00 & 0.15 & \red{0.01} & 1.00 &  \\ 
		% Dec & 1.00 & 1.00 & 1.00 & 1.00 & 1.00 & 1.00 & 1.00 & 1.00 & 1.00 & 1.00 & 0.93 \\ 
		\bottomrule
	\end{tabular}
    \caption{Pairwise Wilcoxon $T$-test for \textit{Month} and \textit{Maximal Spatial Extent}, see \cref{tbl:wilcoxon_baysis_effector_Month_SMax_complete} for complete table}
    \label{tbl:wilcoxon_baysis_effector_Month_SMax}
\end{table}
The table shows, that only the groups of Nov differs significantly from Oct.
% #### START: Table and Plot
\begin{figure}[ht!]
	\centering
	\begin{minipage}{0.5\textwidth}
		\tiny
		\setlength{\tabcolsep}{4pt}
		\centering
		\begin{tabular}{c|c|c|c|c|c|c|c}
			\toprule
			Group & $n$ & $\bar{x}$ & $\sigma$ & $\tilde{x}$ & $min$ & $max$ & $\Delta$ \\
			\midrule
			Jan & 39  & 15204.44 & 10633.53 & 11151.0 & 1772 & 38867 & 37095 \\ 
			Feb & 38  & 13921.39 & 9612.90  & 12118.5 & 1176 & 44548 & 43372 \\ 
			Mar & 51  & 12295.18 & 9266.62  & 9669.0  & 1206 & 48278 & 47072 \\ 
			Apr & 57  & 15190.47 & 11402.17 & 12433.0 & 1415 & 43070 & 41655 \\ 
			May & 50  & 13361.66 & 9244.96  & 10112.5 & 1315 & 40805 & 39490 \\ 
			Jun & 56  & 13241.27 & 8206.12  & 11594.0 & 2632 & 39685 & 37053 \\ 
			Jul & 112 & 17045.40 & 11159.03 & 14569.5 & 1036 & 49765 & 48729 \\ 
			Aug & 88  & 14304.86 & 9246.10  & 11801.5 & 2500 & 42393 & 39893 \\ 
			Sep & 82  & 16700.20 & 10632.10 & 16063.5 & 2006 & 46328 & 44322 \\ 
			Oct & 64  & 12392.00 & 9137.99  & 9882.5  & 1524 & 35354 & 33830 \\ 
			Nov & 56  & 10844.89 & 8940.48  & 8712.0  & 1883 & 36151 & 34268 \\ 
			Dec & 49  & 15302.61 & 11066.27 & 11386.0 & 1925 & 43244 & 41319 \\ 
			\bottomrule
			% \bar{x} - sum = 169803.37, mean = 14150.36
			% \sigma - sum = 118548.27, mean = 9879.02
			% \tilde{x}
		\end{tabular}
		\subcaption[second caption.]{Table of all descriptives}\label{tbl:descriptives_baysis_effector_Month_SMax}
	\end{minipage}%
	\begin{minipage}{0.55\textwidth}
		\pgfplotstableread[col sep=comma]{
			month, count, mean, median, sd, min, max, delta
			Jan , 39  , 15204.44 , 10633.53 , 11151.0 , 1772 , 38867 , 37095 
			Feb , 38  , 13921.39 , 9612.90  , 12118.5 , 1176 , 44548 , 43372 
			Mar , 51  , 12295.18 , 9266.62  , 9669.0  , 1206 , 48278 , 47072 
			Apr , 57  , 15190.47 , 11402.17 , 12433.0 , 1415 , 43070 , 41655 
			May , 50  , 13361.66 , 9244.96  , 10112.5 , 1315 , 40805 , 39490 
			Jun , 56  , 13241.27 , 8206.12  , 11594.0 , 2632 , 39685 , 37053 
			Jul , 112 , 17045.40 , 11159.03 , 14569.5 , 1036 , 49765 , 48729 
			Aug , 88  , 14304.86 , 9246.10  , 11801.5 , 2500 , 42393 , 39893 
			Sep , 82  , 16700.20 , 10632.10 , 16063.5 , 2006 , 46328 , 44322 
			Oct , 64  , 12392.00 , 9137.99  , 9882.5  , 1524 , 35354 , 33830 
			Nov , 56  , 10844.89 , 8940.48  , 8712.0  , 1883 , 36151 , 34268 
			Dec , 49  , 15302.61 , 11066.27 , 11386.0 , 1925 , 43244 , 41319 
		}\data
		\pgfplotstablesort[sort key=mean, sort cmp=float >]{\datasorted}{\data}
		\tiny
		\centering
		\descplotfigwithavg{\datasorted}{14150}{9879}{11}{5}
		\subcaption[second caption.]{Plot of descriptives $\bar{x}$, $\sigma$ and $\tilde{x}$}\label{fig:descriptives_baysis_effector_Month_SMax}
	\end{minipage}%
	\caption{Group descriptives of \textit{Month} and \textit{Maximal Spatial Extent} (Jam Effector)}
	%\vspace{-8mm}
\end{figure}
% #### END: Table and Plot
The descriptives in \cref{tbl:descriptives_baysis_effector_Month_SMax} show that the count $n$ is distributed over the year with a peek in July and low in January. The November shows the shorts spatial length when the July show the longest length in $\bar{x}$ and $\sigma$. There is no distinctive general trend.

% ####################################################
\groupintrosig{Month}{SAvg}{0.0348}{baysis}{effector} 
Although the Kruskal-Wallis test shows significant differences, the Wilcoxon table shows that the differences are not group specific.
% #### START: Table and Plot
\begin{figure}[ht!]
	\centering
	\begin{minipage}{0.5\textwidth}
		\tiny
		\setlength{\tabcolsep}{4pt}
		\centering
		\begin{tabular}{c|c|c|c|c|c|c|c}
			\toprule
			Group & $n$ & $\bar{x}$ & $\sigma$ & $\tilde{x}$ & $min$ & $max$ & $\Delta$ \\
			\midrule
			Jan & 39  & 4600.31 & 3223.30 & 4111.0 & 829  & 14785 & 13956 \\ 
			Feb & 38  & 4046.26 & 2217.96 & 3833.5 & 853  & 84260 & 7573  \\ 
			Mar & 51  & 3803.96 & 2410.63 & 2944.0 & 747  & 10494 & 9747  \\ 
			Apr & 57  & 3990.09 & 2120.45 & 3865.0 & 670  & 10320 & 9650  \\ 
			May & 50  & 4087.96 & 2608.56 & 3552.5 & 393  & 10614 & 10221 \\ 
			Jun & 56  & 4270.54 & 2559.92 & 3895.0 & 779  & 11206 & 10427 \\ 
			Jul & 112 & 4412.27 & 2550.18 & 3826.5 & 784  & 15132 & 14348 \\ 
			Aug & 88  & 4386.16 & 2603.28 & 3533.0 & 1036 & 13744 & 12708 \\ 
			Sep & 82  & 4925.18 & 2776.80 & 4234.0 & 786  & 13605 & 12819 \\ 
			Oct & 64  & 4026.47 & 2211.68 & 3865.5 & 358  & 8116  & 7758  \\ 
			Nov & 56  & 3591.02 & 2391.39 & 2976.5 & 660  & 11167 & 10507 \\ 
			Dec & 49  & 5380.88 & 3286.83 & 4804.0 & 1006 & 17805 & 16799 \\ 
			\bottomrule
			% \bar{x} - sum = 51521.1, mean = 4293.42
			% \sigma - sum = 30960.98, mean = 2580.08
			% \tilde{x}
		\end{tabular}
		\subcaption[second caption.]{Table of all descriptives}\label{tbl:descriptives_baysis_effector_Month_SAvg}
	\end{minipage}%
	\begin{minipage}{0.55\textwidth}
		\pgfplotstableread[col sep=comma]{
			month, count, mean, median, sd, min, max, delta
			Jan , 39  , 4600.31 , 3223.30 , 4111.0 , 829  , 14785 , 13956 
			Feb , 38  , 4046.26 , 2217.96 , 3833.5 , 853  , 84260 , 7573  
			Mar , 51  , 3803.96 , 2410.63 , 2944.0 , 747  , 10494 , 9747  
			Apr , 57  , 3990.09 , 2120.45 , 3865.0 , 670  , 10320 , 9650  
			May , 50  , 4087.96 , 2608.56 , 3552.5 , 393  , 10614 , 10221 
			Jun , 56  , 4270.54 , 2559.92 , 3895.0 , 779  , 11206 , 10427 
			Jul , 112 , 4412.27 , 2550.18 , 3826.5 , 784  , 15132 , 14348 
			Aug , 88  , 4386.16 , 2603.28 , 3533.0 , 1036 , 13744 , 12708 
			Sep , 82  , 4925.18 , 2776.80 , 4234.0 , 786  , 13605 , 12819 
			Oct , 64  , 4026.47 , 2211.68 , 3865.5 , 358  , 8116  , 7758  
			Nov , 56  , 3591.02 , 2391.39 , 2976.5 , 660  , 11167 , 10507 
			Dec , 49  , 5380.88 , 3286.83 , 4804.0 , 1006 , 17805 , 16799 
		}\data
		\pgfplotstablesort[sort key=mean, sort cmp=float >]{\datasorted}{\data}
		\tiny
		\centering
		\descplotfigwithavg{\datasorted}{4293}{2580}{11}{5}
		\subcaption[second caption.]{Plot of descriptives $\bar{x}$, $\sigma$ and $\tilde{x}$}\label{fig:descriptives_baysis_effector_Month_SAvg}
	\end{minipage}%
	\caption{Group descriptives of \textit{Month} and \textit{Average Spatial Extent} (Jam Effector)}
	%\vspace{-8mm}
\end{figure}
% #### END: Table and Plot
The descriptives from \cref{tbl:descriptives_baysis_effector_Month_SAvg} show that November has the shortest average spatial lengths when the December show the longest average lengths in $\bar{x}$ and $\sigma$. There is no distinctive general trend.