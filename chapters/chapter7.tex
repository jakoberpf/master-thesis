\chapter{Summary of analysis}
\label{analysis_summary}
In the previous chapter the datasets where analyzes for collections and the relevance of the correlation was evaluated. The analysis has presented are a number of separated statistical statements and interpretations. Because these are to specific on their own they need to be summarizes to comprise a general statement. This summary and general interpretation is topic of this chapter.

\section{Accidents}
The analysis of the congestion -  accidents correlations is separated into sections, to account for the different accident types of Initiator, Effector and Follower (see \cref{methodology_data_processing}). In this section, the findings for each variable from the different categories are collected, summarizes and interpreted. 

\subsection{Street (\textit{Strasse})}
\label{analysis_sum_Strasse}
The relation classes of \text{Jam Intiator} and \text{Jam Follower} don't show any relevance in the found collections. In the case of the \text{Jam Intiator} this it especially unfortunate, since it is the most promising group for predictions. The relations in the complete matched dataset and the relation class of \text{Jam Effector} are summarizes in the following.

\textbf{Duration} : The maximum and average jam duration is generally related to the street, but not all streets show statistical significant differences. Therefore there is no general significant trend to be interpreted but since the relation has a general significance and individual significant differences are present it can be assumed, that the descriptives are generally representable. When comparing all descriptive means (shown in \cref{fig:baysis_summary_Str_duration_barplot}) it becomes clear, that the maximum and average variables differ by high ranges, which is to be expected, but also show a similar tend.
\begin{figure}[ht!]
    \pgfplotstableread[col sep=comma]{
        Road, gMeanTMax, gMeanTAvg, eMeanTMax, eMeanTAvg, means
        A3  , 225.76   , 89.66    , 297.62   , 104.05 , 179
        A6  , 153.05   , 69.94    , 236.59   , 82.43  , 136
        A9  , 170.85   , 72.92    , 243.45   , 91.31  , 145
        A70 , 106.55   , 50.10    ,          ,        , 78
        A96 , 118.32   , 61.37    , 155.85   , 74.39  , 102
        A7  , 153.37   , 86.55    , 254.66   , 143.74 , 160
        A73 , 125.95   , 54.78    , 157.21   , 47.63  , 96
        A99 , 169.09   , 58.97    , 212.57   , 65.73  , 127
        A92 , 103.86   , 55.24    , 134.71   , 64.52  , 90
        A93 , 163.57   , 82.33    , 212.00   , 104.83 , 141
        A94 , 101.59   , 49.86    , 109.71   , 45.93  , 77
    }\data
    \pgfplotstablesort[sort key=means, sort cmp=float >]{\datasorted}{\data}
    \tiny
    \centering
    \barplotquadwithmeans{\datasorted}{$\bar{x}_{TMax,global}$}{$\bar{x}_{TAvg,global}$}{$\bar{x}_{TMax,effector}$}{$\bar{x}_{TAvg,effector}$}{121}
    \caption{Comparison of descriptives $\bar{x}_{TMax,global}$, $\bar{x}_{TAvg,global}$, $\bar{x}_{TMax,effector}$ and $\bar{x}_{TAvg,effector}$}
    \label{fig:baysis_summary_Str_duration_barplot}
    %\vspace{-8mm}
\end{figure}
The comparison of the global and \text{Jam Effectors} variables presents an general increase of means from the global to \text{Jam Effectors} variable. This means that accidents during a congestion are associated with (temporal) longer jams, than jams in general.

\textbf{Length} : The maximum and average jam length is generally related to the street, but not all streets show statistical significant differences. Therefore there is no general significant trend to be interpreted but since the relation has a general significance and individual significant differences are present it can be assumed, that the descriptives are generally representable. When comparing all descriptive means (shown in \cref{fig:baysis_summary_Str_length_barplot}) it becomes clear, that the maximum and average variables differ by high ranges, which is to be expected, but also show a similar tend.
\begin{figure}[ht!]
    \pgfplotstableread[col sep=comma]{
        Road, meanSMax, meanSAvg, eMeanTMax, means
        A3  , 13874.96 , 4537.56 , 17755.15 , 12056
        A6  , 11067.98 , 4361.50 , 16711.43 , 10714
        A9  , 10680.48 , 4187.15 , 13271.42 , 9380
        A70 , 6676.39  , 3010.45 ,          , 4843
        A96 , 8551.75  , 3678.39 , 11825.98 , 8019
        A7  , 9018.27  , 4141.68 , 11825.00 , 8328
        A73 , 6502.88  , 2683.97 , 7964.98  , 5717
        A99 , 13244.02 , 3240.97 , 17558.70 , 11348
        A92 , 6186.80  , 2926.15 , 7255.76  , 5456
        A93 , 6765.00  , 2525.43 , 8158.42  , 5816
        A94 , 6220.38  , 2691.68 , 7277.64  , 5397
    }\data
    \pgfplotstablesort[sort key=means, sort cmp=float >]{\datasorted}{\data}
    \tiny
    \centering
    \barplottriplewithmeans{\datasorted}{$\bar{x}_{SMax,global}$}{$\bar{x}_{SAvg,global}$}{$\bar{x}_{SMax,effector}$}{7916}
    \caption{Comparison of descriptives $\bar{x}_{SMax,global}$, $\bar{x}_{SAvg,global}$ and $\bar{x}_{SMax,effector}$}
    \label{fig:baysis_summary_Str_length_barplot}
    %\vspace{-8mm}
\end{figure}
The comparison of the global and \text{Jam Effectors} variables presents like with the jam duration an general increase of means from the global to \text{Jam Effectors} variable. This means that accidents during a congestion are associated with (spatial) longer jams, than jams in general.

\textbf{Coverage} The jam coverage is generally related to the street, with most road showing statistical significant differences. Therefore it can be assumed, that the descriptives are generally representable. The comparison of the global and \text{Jam Effectors} variables (shown in \cref{fig:baysis_summary_Str_coverage_barplot}) presents an general decrease of means from the global to \text{Jam Effectors} variable. This means that accidents during a congestion are associated with less denser jams, than jams in general.
\begin{figure}[ht!]
    \pgfplotstableread[col sep=comma]{
        Road, meanCov, eMeanCov, means
        A3  , 38.44 , 32.02 , 35 
        A6  , 46.99 , 30.81 , 39 
        A9  , 43.93 , 36.09 , 40 
        % A70 , 50.29 ,       , 25  
        A96 , 51.94 , 42.90 , 47 
        A7  , 53.46 , 45.60 , 50  
        A73 , 46.49 , 32.94 , 40 
        A99 , 33.21 , 26.79 , 30 
        A92 , 53.15 , 47.71 , 50 
        A93 , 40.81 , 37.25 , 39  
        A94 , 47.76 , 36.79 , 42  
    }\data
    \pgfplotstablesort[sort key=means, sort cmp=float >]{\datasorted}{\data}
    \tiny
    \centering
    \barplotdoublewithmeans{\datasorted}{$\bar{x}_{Cov,global}$}{$\bar{x}_{Cov,effector}$}{40}
    \caption{Comparison of descriptives $\bar{x}_{Cov,global}$ and $\bar{x}_{Cov,effector}$}
    \label{fig:baysis_summary_Str_coverage_barplot}
    %\vspace{-8mm}
\end{figure}
In accordance with these finding, it can be interpreted that in general the street plays a significant role defining the size and form of a congestion. This is not surprising since the demand on the street, which varies heavily between the streets is the major factor influencing the buildup of a congestion. Non the less it is interesting the the differences are substantial with ranges up to 60\,\% increases between groups and are replicated over difference variable relations. Another effect it the increased jam duration and length with the \text{Jam Effectors} in comparison with jams in general. This is due to either the higher probability of a accident in very long jams or that accident during a can jam increase the jam extends itself.

\subsection{Accident category (\textit{Kat})}
\label{analysis_sum_Kat}
The variable \textit{Kat} showed a general dependence to the maximum and average duration and a general trend of increasing duration with the injury gravity of the accident in the global and \textit{Jam Initiators} dataset. In the global dataset the category of accidents with property damage does not fit into this trend and sits in-between the lightly and heavily injured category. This changes when considering just the \textit{Jam Initiators}, where all categories follow one increasing trend. When comparing all descriptive means (shown in \cref{fig:baysis_summary_Kat_duration_barplot}) it becomes clear, that the maximum and average variables differ by high ranges, which is to be expected. The comparison also show that that the jam duration significant increases with the gravity of the accident. The category of deadly accident differs by 130\,min from accidents of lightly injured and property damage accidents. 
\begin{figure}[ht!]
    \pgfplotstableread[col sep=comma]{
        Kat, gMeanTMax, gMeanTAvg, iMeanTMax, iMeanTAvg, means
        1 , 317.67 , 156.06 , 290.07 , 148.76 , 228
        2 , 189.03 , 88.31  , 156.23 , 80.91  , 129
        3 , 155.26 , 67.32  , 120.18 , 63.75  , 102
        7 , 179.35 , 74.08  , 103.13 , 53.61  , 103
    }\data
    \pgfplotstablesort[sort key=means, sort cmp=float >]{\datasorted}{\data}
    \tiny
    \centering
    \barplotquadwithmeans{\datasorted}{$\bar{x}_{TMax,global}$}{$\bar{x}_{TAvg,global}$}{$\bar{x}_{TMax,initiator}$}{$\bar{x}_{TAvg,initiator}$}{140}
    \caption{Comparison of descriptives $\bar{x}_{TMax,global}$, $\bar{x}_{TAvg,global}$, $\bar{x}_{TMax,initiator}$ and $\bar{x}_{TAvg,initiator}$}
    \label{fig:baysis_summary_Kat_duration_barplot}
    %\vspace{-8mm}
\end{figure}
Unlike the temporal length, the spatial length does not correlate with the accident category. According to the global dataset, the temporal distance is heavily related to the accident category, but the descriptives only show a maximum differences of 6\,min, which is arguably low for an interpretation. Non the less in can be stated, that the temporal distance between accident and congestion increased significantly with the injury gravity. The variable also correlates in other datasets, but without significant differences.

\subsection{Accident type (\textit{Typ})}
\label{analysis_sum_Typ}
The \textit{Typ} only significantly relates to the temporal distance between the accident and congestion. In the global dataset accidents of the kind \textit{Driving accident} and \textit{Other} (average of 8.5\,min) have a longer temporal distance than crossing accident or accidents in straight traffic (average of 3.5\,min). A significant differences of similar kinds can be also observed in the \text{Jam Intiator} dataset with a temporal distance of \textit{driving} and \textit{other} accidents (average of 12\,min) compared to \textit{merging}, \textit{crossing} and \textit{straight traffic} accidents (average of 8\,min). The variable also correlates with other variables, but without significant differences. When comparing all descriptive means (shown in \cref{fig:baysis_summary_Typ_TDist_barplot}) it becomes clear, that the temporal distance increases for \textit{Jam Initiators}
and the trend shown in both dataset is mirrored.
\begin{figure}[ht!]
    \pgfplotstableread[col sep=comma]{
        Typ, gMeanTDist, iMeanTDist, means
        1 , 8.33 , 11.40 , 9.9
        3 , 2.91 , 7.67  , 5.3
        6 , 4.78 , 9.21  , 7.0
        7 , 8.70 , 12.27 , 10.5
    }\data
    \pgfplotstablesort[sort key=means, sort cmp=float >]{\datasorted}{\data}
    \tiny
    \centering
    \barplotdoublewithmeans{\datasorted}{$\bar{x}_{TDist,global}$}{$\bar{x}_{TDist,initiator}$}{8.2}
    \caption{Comparison of descriptives $\bar{x}_{TDist,global}$ and $\bar{x}_{TDist,initiator}$}
    \label{fig:baysis_summary_Typ_TDist_barplot}
    %\vspace{-8mm}
\end{figure}

\subsection{Accident kind (\textit{UArt})}
\label{analysis_sum_UArt}
The global dataset shows correlations of the accident kind and the temporal distance as well as the coverage, which are both significant. Collisions with \textit{starting}, \textit{standing}, \textit{stopping}, \textit{ahead and waiting vehicle} and \textit{vehicle on separate lane in same direction} vehicles take an average of 5\,min to form jams, when accident collisions with \textit{obstacles} or \textit{left/right} nearby vehicles take an average of 8\,min to form a congestion. Accidents of the category \textit{turning} and \textit{crossing} vehicles have the most immediate reaction of 2.5\,min. The \textit{Jam Initiator} dataset show the same features of accident collisions with \textit{starting}, \textit{standing}, \textit{stopping}, \textit{turning}, \textit{crossing}, \textit{ahead and waiting vehicle} and \textit{vehicle on separate lane in same direction} vehicles having an average of 9\,min and accident collisions with \textit{obstacles} or \textit{left/right} nearby vehicles an average of 12\,min. This can be also seen in \cref{fig:baysis_summary_UArt1_TDist_barplot} where the two dataset variable are compared directly.
\begin{figure}[ht!]
    \pgfplotstableread[col sep=comma]{
        Typ, gMeanTDist, iMeanTDist, means
        0 , 5.30  , 11.04 , 8.2
        1 , 4.44  , 8.94  , 6.7
        2 , 5.05  , 9.21  , 7.1
        3 , 4.53  , 8.80  , 6.7
        5 , 2.44  , 8.13  , 5.3
        7 , 12.63 , 15.26 , 13.9
        8 , 8.79  , 11.81 , 10.3
        9 , 9.10  , 11.38 , 10.2
    }\data
    \pgfplotstablesort[sort key=means, sort cmp=float >]{\datasorted}{\data}
    \tiny
    \centering
    \barplotdoublewithmeans{\datasorted}{$\bar{x}_{TDist,global}$}{$\bar{x}_{TDist,initiator}$}{8.6}
    \caption{Comparison of descriptives $\bar{x}_{TDist,global}$ and $\bar{x}_{TDist,initiator}$}
    \label{fig:baysis_summary_UArt1_TDist_barplot}
    %\vspace{-8mm}
\end{figure}
The coverage relation shows the same grouping like the temporal distance. Accident collisions with \textit{starting}, \textit{standing}, \textit{stopping}, \textit{ahead and waiting vehicle} and \textit{vehicle on separate lane in same direction} vehicles have an average coverage of 41\,\% and are therefore associated with less dense jams than accident collisions with \textit{obstacles} or \textit{left/right} nearby vehicles, which have a average coverage of 54\,\%. Jams associated with accidents of the category \textit{turning} and \textit{crossing} vehicles are the least dense with 33\,\%. The \textit{Jam Initiator} dataset presents the same trend and supports the findings, but without any significant groups. In the \textit{Jam Initiator} dataset \textit{UArt} also correlates significantly with the maximal temporal extend but only in the form that accidents with vehicles on the \textit{separate lane in the same direction} can be associated with 24\,min longer jams, than accidents with \textit{obstacles} in general.

\subsection{Accident cause (\textit{AUrs})}
\label{analysis_sum_AUrs}
The accident cause describing variable \textit{AUrs} is related to the average spatial extend, temporal distance and coverage, but without significant group differences. The descriptives show minor features like that the jam coverage decreases with the accident causes \textit{Slippery street due to rain} to \textit{Cart track due to rain, snow or ice} to by 13\,\% on average in the \textit{Jam Initiator} dataset. The global dataset show the same feature with an average decrease of 17\,\%. The global dataset they also reveals that the (spatial) jam length increases with the accident causes \textit{Slippery street due to rain} over \textit{Visibility issues due to sun or glare} to \textit{Slippery street due to snow or ice} by 2382\,m on average. The temporal distance between accident and congestion decreases with the accident causes \textit{Slippery street due to snow or ice} and \textit{Slippery street due to rain} (average of 7.7\,min) to \textit{Visibility issues due to sun or glare} to by 3\,min on average.

\subsection{Accident collision object type (\textit{AufHi})}
\label{analysis_sum_AufHi}
The variable \textit{AufHi} describes the type of collision object and is related to the temporal distance and coverage in the global dataset. The temporal distance between jams and accidents is 4.7\,min for \textit{trees} and 8.4\,min for \textit{guardrail} or \textit{other obstacles}. The jam coverage of collisions with \textit{trees} (40\,\% coverage) is 10\,\% lower than the 50\,\% coverage of the groups of \textit{guardrail} or \textit{other obstacles}. The variable also correlates with other variables, but without meaningful interpretation or significance.

\subsection{Accident environment characteristic (\textit{Char})}
\label{analysis_sum_Char}
The accident characteristic variable correlates significantly with the temporal distance in the \textit{Jam Initiator} dataset, but without groups specific difference.

\subsection{Accident lighting environment (\textit{Lich})}
\label{analysis_sum_Lich}
The lighting situation correlates significantly with the coverage and shows that jams in darkness are 18\,\% denser than jams in daylight. The state of the street lighting show a similar effect of 10\,\% denser jams in case of broken street lighting.

\subsection{Road condition (\textit{Zust})}
\label{analysis_sum_Zust}
The road condition correlates with the coverage of a jam, in the way that the coverage increases by 20\,\%-25\,\% from \textit{dry} over \textit{wet} to \textit{ice}.

\subsection{Weekday (\textit{WoTag})}
\label{analysis_sum_WoTag}
The week day correlates with the spatial and temporal extends as well as the temporal distance, coverage and time loss in multiple datasets. But only the the coverage and time loss (HGV) correlation show significant differences. The coverage of jams on Monday's, Saturday's and Sunday's is 7\,\%-8\,\% higher than on other week days. The variable also correlates with other variables, but without meaningful interpretation or significance.

\subsection{Month (\textit{Month})}
\label{analysis_sum_Month}
The month of the accident correlates in multiple datasets and with most congestion variables, but only in the \textit{Jam Effector} the maximal and average spatial extend show significant differences. The months of January and November have the shortest extends when July and December have the longest. Together with the other months they do not form a distinctive trend. 

\section{Roadworks}
Unlike the analysis of the congestion - accident correlations the roadwork correlations are not separated into global and \textit{Jam Initiator}. Although as stated in the methodology \cref{methodology_data_processing} a classification of congestion - roadwork matched would be viable to analysis just congestion which are allocated spatial and temporal in from of a roadwork. The analysis was finished just in time of finishing the thesis. However this did not leave time for implementing this analysis in the textual part of the thesis, yet the results of the analysis are available in the repository linked in the introduction.

\subsection{Street (\textit{Strasse})}
The road of the roadwork correlates strongly and significantly with all of the congestion characteristics, due to the very high number of samples. The distribution of the jam duration by the road is shown in \cref{fig:arbis_summary_Str_temporal} and show a distinctive trend.
\begin{figure}[ht!]
	\pgfplotstableread[col sep=comma]{
		Road, gMeanTMax, gMeanTAvg, means
		A9   , 85.61  , 159.55 , 123
		A7   , 87.76  , 158.38 , 123
		A70  , 154.38 , 228.90 , 192
		A71  , 64.90  , 83.40  , 74
		A6   , 65.60  , 142.24 , 104
		A73  , 56.23  , 153.59 , 105
		A3   , 89.94  , 233.66 , 162
		A99  , 45.38  , 158.88 , 102
		A96  , 69.31  , 105.80 , 88
		A995 , 37.14  , 60.00  , 49
		A92  , 81.83  , 132.40 , 107
		A93  , 115.96 , 157.33 , 137
		A94  , 84.98  , 179.57 , 132  
	}\data
	\pgfplotstablesort[sort key=means, sort cmp=float >]{\datasorted}{\data}
	\tiny
	\centering
	\barplotdoublewithmeans{\datasorted}{$\bar{x}_{TMax}$}{$\bar{x}_{TAvg}$}{115}
	\caption{Comparison of descriptives $\bar{x}_{TMax}$ and $\bar{x}_{TAvg}$ (\textit{TMax/TAvg} by \textit{Street})}
	\label{fig:arbis_summary_Str_temporal}
	%\vspace{-8mm}
\end{figure}
The diagram is sorted by the mean of maximal and average duration and therefore shows that the A70 and A3 have considerable longer jams and the A96, A71 and A995 considerable shorter jams in comparison to the mean. The distribution of the jam length by the road is shown in \cref{fig:arbis_summary_Str_spatial} and also shows a distinctive trend.
\begin{figure}[ht!]
	\pgfplotstableread[col sep=comma]{
		Road, gMeanSMax, gMeanSAvg, means
		A9   , 3537.87 , 8655.57  , 6097
		A7   , 2917.97 , 6238.63  , 4578
		A70  , 3170.10 , 5729.56  , 4450
		A71  , 2572.50 , 3899.00  , 3236
		A6   , 3189.79 , 8421.67  , 5806
		A73  , 2508.40 , 7717.21  , 5113
		A3   , 3733.90 , 11836.50 , 7785
		A99  , 2381.29 , 9337.49  , 5859
		A96  , 2488.22 , 5639.41  , 4064
		A995 , 1976.93 , 3618.14  , 2798
		A92  , 3360.88 , 5758.10  , 4559
		A93  , 2243.26 , 3530.39  , 2887
		A94  , 2557.62 , 5849.05  , 4203
	}\data
	\pgfplotstablesort[sort key=means, sort cmp=float >]{\datasorted}{\data}
	\tiny
	\centering
	\barplotdoublewithmeans{\datasorted}{$\bar{x}_{SMax}$}{$\bar{x}_{SAvg}$}{4726}
	\caption{Comparison of descriptives $\bar{x}_{SMax}$ and $\bar{x}_{SAvg}$ (\textit{SMax/SAvg} by \textit{Street})}
	\label{fig:arbis_summary_Str_spatial}
	%\vspace{-8mm}
\end{figure}
The diagram is sort by the mean of maximal and average length like with the duration diagram. It shows that the A3, A9, A99 and A6 have considerable longer jams in comparison to the mean. The A71, A93 and A995 have considerable shorter jams in comparison to the mean.

The A9 and A96 have a significantly higher temporal distance than the A3, A7 and A92, but the difference is only about 2\,min. The general trend of the temporal distance shows that the A9, A71, A73 and A96 have higher distances, with a maximum deviation of 2\,min from the overall $\bar{x}$.

The spatial distance can be interpreted that the roads A94 and A99 have a higher distance (302\,m on average) than the A73 and A93. The general trend also that the A94, A995, A99 and A9 have considerable higher and the A70, A73, A93, A71 considerable lower distances when compared to the overall $\bar{x}$ of 145\,m.

For the coverage it can be interpreted that the road A70, A92, A93, and A96 have a significantly higher coverage than the A3, A73 and A99. The time-loss of cars and heavy goods vehicles correlate significantly, but don't provide interpretable differences.

\subsection{Month (\textit{Month})}
The month of the roadwork correlates strongly with most of the congestion characteristics.

\section{Predictability ??}

