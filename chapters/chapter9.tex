\chapter{Conclusion}
In conclusion it can be generally stated that there are significant correlations between congestion and incident characteristics, but with many limitations. The summary of the analysis in \cref{analysis_summary} showed, that general characteristics like Month and Location (Road) are considerably related to congestion characteristics like duration, length and 

\section{Answer of research question}

In accodance of the summary and conlustion the reserach wquation of ?? can be ansered with yes, they do. But at the same time it need to be said that most of the correlation are not reliable and generla enought to be used for prdic

\section{Usability of results and interpretation}

\todo{Are the results usefull?}

\section{Future development and possible improvements} 

\todo{What could be improved and what could be developed?}

\subsubsection{intraclass correlation coefficient (ICC)}
Usage of ICC instead of Eta and Kruskal-Wallis H.
% https://stats.stackexchange.com/questions/73065/correlation-coefficient-between-a-non-dichotomous-nominal-variable-and-a-numer
% https://pingouin-stats.org/generated/pingouin.intraclass_corr.html

\subsubsection{Improve DBSCAN}
The improved DBSCAN algorithm based on cell-like P systems with promoters and inhibitors
% https://journals.plos.org/plosone/article?id=10.1371/journal.pone.0200751

\subsubsection{Clustering in 3D (including cell speed)}
Expand DBSCAN to three dimension from space and time (2D) to include cell speed into clustering algorithm.