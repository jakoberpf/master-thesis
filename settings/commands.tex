%\newcommand{\code}[1]{\colorbox{lightgray}{\texttt{#1}}}
%\lstinline{snippet}
%http://tex.stackexchange.com/questions/65291/code-snippet-in-text

% This "\code{my code}" can be used to highlight small code snippets in the text like names of variables or methods.
\newcommand{\code}[1]{\textcolor{black}{\texttt{#1}}}

% The "\todo{this still has to be done}" is a command that highlights todos in the text.
\newcommand{\todo}[1]{\textcolor{vscodered}{TODO: \texttt{#1}}}

\newcommand{\red}[1]{\textcolor{red}{#1}}

\usepackage{textcomp}
\newcommand{\textapprox}{\raisebox{0.5ex}{\texttildelow}}

\newcommand{\centerheading}[1]{
    \subsubsection{#1}
    \vspace{-4mm}
}

%% Text snippets
% Accident/Roadwork Section Intro 
\newcommand{\secintroend}[2]{In the following sections, the correlated relations of the variables in \cref{tbl:correlation_list_#1_#2} are analyzed and an initial interpretation of each significant correlation is introduced. Groups with an insufficient sample size (see \cref{correlation_uncertainty} are neglected and not shown to reduce the overall table size. The descriptive tables, showing the count ($n$), mean ($\bar{x}$), standard deviation ($\sigma$), median ($\tilde{x}$), $min$, $max$ and range ($\Delta$) therefore only contain groups with significant sample sizes.}

% Variable Analysis Intro with uncertainty
\newcommand{\varintrosimplewithsam}[1]{This section analyzes the correlated relations of the accident variable \textit{#1}. Groups with an insufficient sample size (see \cref{correlation_uncertainty} are neglected and not considered.}
% Variable Analysis Intro without uncertainty
\newcommand{\varintrosimplewithoutsam}[1]{This section analyzes the correlated relations of the accident variable \textit{#1}.}

% Variable Analysis Intro with uncertainty
\newcommand{\varintrowithsam}[2]{This section analyzes the correlated relations of the accident variable \textit{#2}. Groups with an insufficient sample size (see \cref{correlation_uncertainty} are neglected and not considered. The encoding and description of the variable \textit{#2} is shown in \cref{tbl:#1_dataset_#2}.}
% Variable Analysis Intro without uncertainty
\newcommand{\varintrowithoutsam}[2]{This section analyzes the correlated relations of the accident variable \textit{#2}. The encoding and description of the variable \textit{#2} is shown in \cref{tbl:#1_dataset_#2}.}

% Variable Analysis Intro when a single relation shows no significance
\newcommand{\varintronosigsing}[2]{The Kruskal-Wallis test of \textit{#1} - \textit{#2} results in a $p$-value above the defined $\alpha$-level. Therefore the null hypothesis can't be rejected and no significant differences in \textit{#2} for the relations of \textit{#1} are present.}
% Variable Analysis Intro when just two relations show no significance
\newcommand{\varintronosigdouble}[2]{Both Kruskal-Wallis tests of #2 result in $p$-values above the defined $\alpha$-level. Therefore the null hypothesis can't be rejected and no significant differences in \textit{#1} for Both relations are present.}
% Variable Analysis Intro when multiple relations show no significance
\newcommand{\varintronosigmul}[2]{The Kruskal-Wallis tests of #2 result in $p$-values above the defined $\alpha$-level. Therefore the null hypothesis can't be rejected and no significant differences in \textit{#1} for these relations are present.}

% Group Analysis Intro with significant alpha and above .0001
\newcommand{\groupintrosig}[5]{The Kruskal-Wallis test of the relation \textit{#1} - \textit{#2} produces a $p$-value of #3, which is below the defined $\alpha$-level. Therefore the null hypothesis can be rejected, meaning that there are significant differences in the variable \textit{#2} between the groups of \textit{#1}. The significant groups can be identified with a pairwise Wilcoxon $T$-test, which is shown in \cref{tbl:wilcoxon_#4_#5_#1_#2}.}
% Group Analysis Intro with significant alpha and below .0001
\newcommand{\groupintrosigsig}[4]{The Kruskal-Wallis test of the relation \textit{#1} - \textit{#2} produces a $p$-value below .0001, which is far below the defined $\alpha$-level. Therefore the null hypothesis can be rejected, meaning that there are strong significant differences in the variable \textit{#2} between the groups of \textit{#1}. The significant groups can be identified with a pairwise Wilcoxon $T$-test, which is shown in \cref{tbl:wilcoxon_#3_#4_#1_#2}.}

%% Graphic snippets

\newcommand{\barplotsingle}[3]{
    \begin{tikzpicture}
        \begin{axis}[
            ybar,
            width=\textwidth,
            height=5cm,
            xmajorgrids=true,
            ymajorgrids=true,
            xtick=data,
            xticklabels from table={#1}{[index]0}
        ] 
        \addplot table[x expr=\coordindex, y=#2] {#1};
        \legend{#3}
        \end{axis}
    \end{tikzpicture}
}

\newcommand{\barplotsinglewithmeans}[3]{
    \begin{tikzpicture}
        \begin{axis}[
            ybar,
            % bar width=.1cm,
            % Gitterlinien für y-Ticks
            width=\textwidth,
            height=5cm,
            % xmajorgrids=true,
            ymajorgrids=true,
            xtick=data,
            xticklabels from table={#1}{[index]0},
            % zusätzliche Beschriftung an y-Achse
            every extra y tick/.style={
                tick0/.initial=teal,
                yticklabel style={
                    color=\pgfkeysvalueof{/pgfplots/tick\ticknum},
                    xshift=-10pt
                },
            },
            extra y ticks={#3},
        ] 
        \addplot table[x expr=\coordindex, y index={1}] {#1};
        \addplot[draw=teal,smooth] table[x expr=\coordindex, y index={2}] {#1};
        \legend{#2}
        \end{axis}
    \end{tikzpicture}
}

\newcommand{\barplotdouble}[5]{
    \begin{tikzpicture}
        \begin{axis}[
            ybar,
            width=\textwidth,
            height=5cm,
            xmajorgrids=true,
            ymajorgrids=true,
            xtick=data,
            xticklabels from table={#1}{[index]0}
        ] 
        \addplot table[x expr=\coordindex, y=#2] {#1};
        \addplot table[x expr=\coordindex, y=#3] {#1};
        \legend{#4,#5}
        \end{axis}
    \end{tikzpicture}
}

\newcommand{\barplotdoublewithmeans}[4]{
    \begin{tikzpicture}
        \begin{axis}[
            ybar,
            % bar width=.1cm,
            % Gitterlinien für y-Ticks
            width=\textwidth,
            height=5cm,
            % xmajorgrids=true,
            ymajorgrids=true,
            xtick=data,
            xticklabels from table={#1}{[index]0},
            % zusätzliche Beschriftung an y-Achse
            every extra y tick/.style={
                tick0/.initial=teal,
                yticklabel style={
                    color=\pgfkeysvalueof{/pgfplots/tick\ticknum},
                    xshift=-10pt
                },
            },
            extra y ticks={#4},
        ] 
        \addplot table[x expr=\coordindex, y index={1}] {#1};
        \addplot table[x expr=\coordindex, y index={2}] {#1};
        \addplot[draw=teal,smooth] table[x expr=\coordindex, y index={3}] {#1};
        \legend{#2,#3}
        \end{axis}
    \end{tikzpicture}
}

\newcommand{\barplottriple}[7]{
    \begin{tikzpicture}
        \begin{axis}[
            ybar,
            bar width=.1cm,
             % Gitterlinien für y-Ticks
            width=\textwidth,
            height=5cm,
            xmajorgrids=true,
            ymajorgrids=true,
            xtick=data,
            xticklabels from table={#1}{[index]0}
        ] 
        \addplot table[x expr=\coordindex, y=#2] {#1};
        \addplot table[x expr=\coordindex, y=#3] {#1};
        \addplot table[x expr=\coordindex, y=#4] {#1};
        \legend{#5,#6,#7}
        \end{axis}
    \end{tikzpicture}
}

\newcommand{\barplottriplewithmeans}[5]{
    \begin{tikzpicture}
        \begin{axis}[
            ybar,
            bar width=.1cm,
             % Gitterlinien für y-Ticks
            width=\textwidth,
            height=5cm,
            % xmajorgrids=true,
            ymajorgrids=true,
            xtick=data,
            xticklabels from table={#1}{[index]0},
            % zusätzliche Beschriftung an y-Achse
            every extra y tick/.style={
                tick0/.initial=teal,
                yticklabel style={
                    color=\pgfkeysvalueof{/pgfplots/tick\ticknum},
                    xshift=-8pt
                },
            },
            extra y ticks={#5},
        ] 
        \addplot table[x expr=\coordindex, y index={1}] {#1};
        \addplot table[x expr=\coordindex, y index={2}] {#1};
        \addplot table[x expr=\coordindex, y index={3}] {#1};
        \addplot[draw=teal,smooth] table[x expr=\coordindex, y index={4}] {#1};
        \legend{#2,#3,#4}
        \end{axis}
    \end{tikzpicture}
}

\newcommand{\barplotquad}[5]{
    \begin{tikzpicture}
        \begin{axis}[
            ybar,
            bar width=.1cm,
            width=\textwidth,
            height=5cm,
            % xmajorgrids=true,
            ymajorgrids=true,
            xtick=data,
            xticklabels from table={#1}{[index]0}
        ] 
        \addplot table[x expr=\coordindex, y index={1}] {#1};
        \addplot table[x expr=\coordindex, y index={2}] {#1};
        \addplot table[x expr=\coordindex, y index={3}] {#1};
        \addplot table[x expr=\coordindex, y index={4}] {#1};
        \legend{#2,#3,#4,#5}
        \end{axis}
    \end{tikzpicture}
}

\newcommand{\barplotquadwithmeans}[6]{
    \begin{tikzpicture}
        \begin{axis}[
            ybar,
            bar width=.1cm,
             % Gitterlinien für y-Ticks
            width=\textwidth,
            height=5cm,
            % xmajorgrids=true,
            ymajorgrids=true,
            xtick=data,
            xticklabels from table={#1}{[index]0},
            % zusätzliche Beschriftung an y-Achse
            every extra y tick/.style={
                tick0/.initial=teal,
                yticklabel style={
                    color=\pgfkeysvalueof{/pgfplots/tick\ticknum},
                    xshift=-8pt
                },
            },
            extra y ticks={#6},
        ] 
        \addplot table[x expr=\coordindex, y index={1}] {#1};
        \addplot table[x expr=\coordindex, y index={2}] {#1};
        \addplot table[x expr=\coordindex, y index={3}] {#1};
        \addplot table[x expr=\coordindex, y index={4}] {#1};
        \addplot[draw=teal,smooth] table[x expr=\coordindex, y index={5}] {#1};
        \legend{#2,#3,#4,#5}
        \end{axis}
    \end{tikzpicture}
}

% Descriptive Line Plot without average
\newcommand{\descplotfigwithoutavg}[3]{
    \begin{tikzpicture}
        \begin{axis}[
            % Gitterlinien für y-Ticks
            width=\textwidth,
            height=#3cm,
            xmajorgrids=true,
            ymajorgrids=true,
            xtick=data,
            xmin=-0.2,xmax=#2 + 0.2,
            xticklabels from table={#1}{[index]0}
        ]
        \addplot table [y expr=(\thisrow{mean}), x expr=\coordindex] {#1};
        \addplot table [y expr=(\thisrow{median}), x expr=\coordindex] {#1};
        \addplot table [y expr=(\thisrow{sd}), x expr=\coordindex] {#1};
        \legend{
            $\bar{x}$,$\sigma$,$\tilde{x}$}
        \end{axis}
    \end{tikzpicture}\vfill
}

% Descriptive Line Plot with average
\newcommand{\descplotfigwithavg}[5]{
    \begin{tikzpicture}
        \begin{axis}[
            % Gitterlinien für y-Ticks
            width=\textwidth,
            height=#5cm,
            xmajorgrids=true,
            ymajorgrids=true,
            xtick=data,
            xmin=-0.2,xmax=#4 + 0.2,
            xticklabels from table={#1}{[index]0},
            % zusätzliche Beschriftung an y-Achse
            every extra y tick/.style={
                tick0/.initial=blue,
                tick1/.initial=red,
                yticklabel style={
                    color=\pgfkeysvalueof{/pgfplots/tick\ticknum},
                    xshift=-8pt
                },
            },
            extra y ticks={#2,#3},
        ]
        \addplot table [y expr=(\thisrow{mean}), x expr=\coordindex] {#1};
        \addplot table [y expr=(\thisrow{median}), x expr=\coordindex] {#1};
        \addplot table [y expr=(\thisrow{sd}), x expr=\coordindex] {#1};
        \legend{
            $\bar{x}$,$\sigma$,$\tilde{x}$}
        \end{axis}
    \end{tikzpicture}\vfill
}

% Descriptive Line Plot with average and custom labels
\newcommand{\descplotfigwithcustomavg}[7]{
    \begin{tikzpicture}
        \begin{axis}[
            % Gitterlinien für y-Ticks
            width=\textwidth,
            height=#7cm,
            xmajorgrids=true,
            ymajorgrids=true,
            xtick=data,
            xmin=-0.2,xmax=#6 + 0.2,
            xticklabels from table={#1}{[index]0},
            % zusätzliche Beschriftung an y-Achse
            every extra y tick/.style={
                tick0/.initial=blue,
                tick1/.initial=red,
                yticklabel style={
                    color=\pgfkeysvalueof{/pgfplots/tick\ticknum},
                    xshift=-8pt
                },
            },
            extra y ticks={#2,#3},
            extra y tick labels={#4,#5}
        ]
        \addplot table [y expr=(\thisrow{mean}), x expr=\coordindex] {#1};
        \addplot table [y expr=(\thisrow{median}), x expr=\coordindex] {#1};
        \addplot table [y expr=(\thisrow{sd}), x expr=\coordindex] {#1};
        \legend{
            $\bar{x}$,$\sigma$,$\tilde{x}$}
        \end{axis}
    \end{tikzpicture}\vfill
}

%% Math snippets

% \newcommand{\maxVal}[1]{
%     \pgfkeys{/pgf/fpu}
%     \pgfmathsetmacro\buffer{0.0}
%     \pgfplotstableforeachcolumnelement{#1}\of\data\as\cellValue{\pgfmathsetmacro{\buffer}{min(\buffer,\cellValue)}}
% }

% \newcommand{\maxVal}[1]{
%     \pgfkeys{/pgf/fpu}
%     \pgfmathsetmacro\buffer{0.0}
%     \pgfplotstableforeachcolumnelement{#1}\of\data\as\cellValue{\pgfmathsetmacro{\buffer}{max(\buffer,\cellValue)}}
% }

% \newcommand\DrawVMean[1][]{
%     \draw[#1] 
%     (axis cs:200,\pgfkeysvalueof{/pgfplots/ymin}) -- (axis cs:200,\pgfkeysvalueof{/pgfplots/ymax});
% }

% \newcommand\DrawHMean[1][]{
%     \draw[#1] 
%     (axis cs:\pgfkeysvalueof{/pgfplots/xmin},\Mean) -- (axis cs:\pgfkeysvalueof{/pgfplots/xmax},\Mean);
% }

% \newcommand{\MakeMeans}[2]{
%   \pgfplotstableset{
%     create on use/new/.style={
%     create col/expr={\pgfmathaccuma + \thisrow{#2}}},
%   }
%   \pgfplotstablegetrowsof{#1}
%   \pgfmathsetmacro{\NumRows}{\pgfplotsretval}
%   \pgfplotstablegetelem{\numexpr\NumRows-1\relax}{new}\of{#1} 
%   \pgfmathsetmacro{\Sum}{\pgfplotsretval}
%   \pgfmathsetmacro{\Mean}{\Sum/\NumRows}
% }
